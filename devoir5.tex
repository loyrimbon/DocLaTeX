\documentclass[a4paper,10pt]{article}

\usepackage[utf8]{inputenc}
\usepackage[T1]{fontenc}
\usepackage[francais]{babel}
\usepackage{amsthm}
\usepackage{amsmath}
\usepackage{graphics}
\usepackage{amssymb}
\usepackage{color}
\usepackage[table,xcdraw]{xcolor}
\usepackage{tikz}
\usepackage[top=2cm,bottom=2cm,left=2cm,right=2cm]{geometry}
%\usepackage{mathsfs}

%\title{Devoir 1 de Mathématiques}
%\author{\textsc{RIMBON} Loy}
%\date{\today}

%!!!!!!!!!!!!!!!!!!!!!!!!!!!!               !!!!!!!!!!!!!!!!!!!!!!!!!!!!
%!!!!!!!!!!!!!!!!!!!!!!!!!!!!   Exercices   !!!!!!!!!!!!!!!!!!!!!!!!!!!!
%!!!!!!!!!!!!!!!!!!!!!!!!!!!!               !!!!!!!!!!!!!!!!!!!!!!!!!!!!

\newcounter{exercice}
\setcounter{exercice}{0}
\newcommand{\exercice}{%
	\refstepcounter{exercice}%
	\bigskip
	\bigskip
	\noindent\textbf{Exercice \theexercice.}~%
}
%%%%%%%%%%%%%%%%%%%%%%%%%%%%%%%%%%%%%%%%%%%%%%%%%%%%%%%
% styles TikZ pour les blocks
\tikzstyle{block}=[draw,rectangle,fill=white,inner sep=0mm,minimum size=12mm]
\tikzstyle{emptyblock}=[rectangle,fill=white,inner sep=0mm,minimum size=12mm]
\tikzstyle{sumg}=[draw,circle,node distance=1cm,inner sep=0mm,minimum size=10mm]
\tikzstyle{sum}=[draw,circle,node distance=1cm,inner sep=0mm,minimum size=6mm]
\tikzstyle{input}=[coordinate]
%%%%%%%%%%%%%%%%%%%%%%%%%%%%%%%%%%%%%%%%%%%%%%%%%%%%%%%
%!!!!!!!!!!!!!!!!!!!                                !!!!!!!!!!!!!!!!!!!
%!!!!!!!!!!!!!!!!!!!   Vos commandes personnelles   !!!!!!!!!!!!!!!!!!!
%!!!!!!!!!!!!!!!!!!!                                !!!!!!!!!!!!!!!!!!!

\newcommand{\stagiaire}{Loy RIMBON}
\newcommand{\datedevoir}{30 septembre 2016}
\newcommand{\titre}{Devoir \no 5}

\newcommand{\R}{\mathbb{R}}
\newcommand{\C}{\mathbb{C}}
\newcommand{\N}{\mathbb{N}}
\newcommand{\Z}{\mathbb{Z}}
\newcommand{\Q}{\mathbb{Q}}
\newcommand{\LCI}{$E\times E\longrightarrow E$ ~}

\renewcommand{\leq}{\leqslant}
\renewcommand{\geq}{\geqslant}


\begin{document}
%\maketitle

\noindent\stagiaire\hfill\datedevoir

\bigskip
\bigskip
\begin{center}
{\large\bfseries\titre}
\end{center}
\bigskip
\bigskip

%%%%%%%%%%%%%%%%%%%%%%%%%%%%%%%%%%%%%%%%%%%%%%%%%%%%%%%%%%%%%%%%%%%%%%%%%%%%%
%%%%%%%%%%%%%%%%%%%%%%%%%%%%%%%%%%%%%%%%%%%%%%%%%%%%%%%%%%%%%%%%%%%%%%%%%%%%%
%%%%%%%%%%%%%%%%%%%%%%%%%%%%%%%%%%%%%%%%%%%%%%%%%%%%%%%%%%%%%%%%%%%%%%%%%%%%%

%%%%%%%%%%%%%%%%%%%%%%%%%%%%%%%%%%%%%%%%%%%%%%%%%%%%%%%%%%%%%%%%%%%%%%%%%%%%%
%%%%%%%%%%%%%%%%%%%%%%%%%%%%%%%%%%%%%%%%%%%%%%%%%%%%%%%%%%%%%%%%%%%%%%%%%%%%%
%%%%%%%%%%%%%%%%%%%%%%%%%%%%%%%%%%%%%%%%%%%%%%%%%%%%%%%%%%%%%%%%%%%%%%%%%%%%%
\exercice 

Soit $(G,\times), \forall a \in G,x\longmapsto^{ \varphi_a}$ un groupe noté multiplicativemen
\begin{enumerate}

%QUESTION 1
\item
	Si $G$ est commutatif.
	$\forall x \in G, \varphi_a(x)=axa^{-1}=aa^{-1}x=x$.
	Ainsi, si $G$ est commutatif l'application $\varphi_a$ est l'application $Id(x)$.
	
\bigskip

%QUESTION 2
\item
Soit $(x,y) \in G^2, x \neq y $
\begin{align}
	\varphi_a(x) \times \varphi_a(y) &= axa^{-1} \times aya^{-1}\notag \\
	&= axya^{-1} \notag \\
	&= \varphi_a(xy) \notag
\end{align}

L'application $\varphi_a$ est un endomorphisme.

$\varphi_a(x)=axa^{-1}  $ et $\varphi_a(y)=aya^{-1}  $

Ainsi,$ x \neq y \implies \varphi_a(x) \neq \varphi_a(y)$, l'application $\varphi_a$ est injective.

L'application $ \varphi_a $ est l'application $Id(x) de G\longrightarrow G$, donc chaque élément de $G$ est sa propre image. Chaque élément de $G$ à donc un antécédent. L'application est donc surjective.

Ainsi, l'application $ \varphi_a $ étant injective et surjective elle est bijective.

On peut donc en déduire que l'application $ \varphi_a $ est un automorphisme de $G$.

\bigskip

%QUESTION 3
\item
Soit $ (a,b) \in G^2$, $\varphi_b \circ\varphi_a=\varphi_b (\varphi_a)$
Pour un élément $x$ de $G$ on à la relation suivante :
$x\longmapsto baxa^{-1}b^{-1}$.


$\forall (a,b,c) \in G^3 $

\begin{align}
	\varphi_a \circ(\varphi_b \circ\varphi_c) &= \varphi_a \circ(\varphi_b(\varphi_c(x))) \notag \\
	&=  \varphi_a \circ(\varphi_{bc}x)\notag \\
	&= \varphi_a(bcxc^{-1}b^{-1})\notag \\
	&= abcxc^{-1}b^{-1}a^{-1}\notag \\
	&= ab(cxc^{-1})b^{-1}a^{-1}\notag \\
	&= ab(\varphi_c(x)b^{-1}a^{-1}\notag \\
	&= \varphi_{ab}(\varphi_c(x)\notag \\
	&= (\varphi_a \circ\varphi_b)\varphi_c(x)\notag 
\end{align}

Ainsi, la loi $\circ$ est associative.


Soit $e_g \in G$, 
\begin{align}
	\varphi_{e_g}(x) &=e_gxe_g^-1 \notag \\
	&= x \notag
\end{align}
Il existe bien un élément neutre pour la loi $\circ$.


\begin{align}
	\varphi_a \circ \varphi_{a^{-1}}&=\varphi_a(a^{-1}x(a^{-1})^{-1})\notag \\
	&=aa^{-1}xaa^{-1} \notag \\
	&= x
\end{align}
La loi $\circ$ dispose bien d'un symétrique $\forall a \in G$.
\end{enumerate}

%%%%%%%%%%%%%%%%%%%%%%%%%%%%%%%%%%%%%%%%%%%%%%%%%%%%%%%%%%%%%%%%%%%%%%%%%%%%%
%%%%%%%%%%%%%%%%%%%%%%%%%%%%%%%%%%%%%%%%%%%%%%%%%%%%%%%%%%%%%%%%%%%%%%%%%%%%%
%%%%%%%%%%%%%%%%%%%%%%%%%%%%%%%%%%%%%%%%%%%%%%%%%%%%%%%%%%%%%%%%%%%%%%%%%%%%%
\exercice 

\begin{enumerate}

%QUESTION 1
\item
\begin{enumerate}(a)
\item \textbf{Dessin des registres.}

Le registre 1 correspond au polynôme caractéristique suivant :
$X^3+X^2+X+1$

\begin{tikzpicture}[auto, node distance=2cm,>=latex,line width=.5pt]
		% définitions de longueurs
		\newlength{\sephonr}
		\setlength{\sephonr}{4cm}
		\newlength{\sepvera}
		\setlength{\sepvera}{2cm}
		\newlength{\sepverb}
		\setlength{\sepverb}{2cm}
		% placement des blocks
		\node[input,name=input]{};
		\node[block,right of=input,node distance=3cm](s0){$1$};
		\node[block,right of=s0,node distance=\sephonr](s1){$1$};
		\node[block,right of=s1,node distance=\sephonr](s2){$1$};
		\node[input,right of=s0,node distance=.5\sephonr](output1){};
		%\node[sum,above of=output1,node distance=\sepverb](c1){$+$};
		\node[input,right of=s1,node distance=.5\sephonr](output2){};
		\node[sum,above of=output1,node distance=\sepverb](c1){$+$};
		\node[sum,above of=output2,node distance=\sepverb](c2){$+$};
		\node[input,right of=s2,node distance=.5\sephonr](output3){};
		\node[input,above of=output3,node distance=\sepverb](output5){};
		\node[input,above of=input,node distance=\sepverb](output6){};
		\node[input,below of=input,node distance=.5\sepverb](output7){};
		% placement des flêches
		\draw[draw,->](s1)--(s0);
		\draw[draw,->](s2)--(s1);
		% première rangée de flèches verticales
		\draw[draw,->](output1)--(c1);
		\draw[draw,->](output2)--(c2);
		% deuxième rangée de flèches verticales
		\draw[draw](input)--(s0);
		\draw[draw](input)--(output6);
		%\draw[draw](output6)--(input);
		\draw[draw,->](output6)--(c1);
		\draw[draw,->](c1)--(c2);
		\draw[draw,->](output3)--(s2);
		\draw[draw](output3)--(output5);
		\draw[draw](c2)--(output5);
		\draw[draw,<-](output7)--(s0.west);
	\end{tikzpicture}


Le registre 2 correspond au polynôme caractéristique suivant :
$X^3+X^2+1$

\begin{tikzpicture}[auto, node distance=2cm,>=latex,line width=.5pt]
			% définitions de longueurs
		
		\setlength{\sephonr}{4cm}
		
		\setlength{\sepvera}{2cm}
		
		\setlength{\sepverb}{2cm}
		% placement des blocks
		\node[input,name=input]{};
		\node[block,right of=input,node distance=3cm](s10){$1$};
		\node[block,right of=s10,node distance=\sephonr](s11){$1$};
		\node[block,right of=s11,node distance=\sephonr](s12){$1$};
		\node[input,right of=s10,node distance=.5\sephonr](output11){};
		%\node[sum,above of=output1,node distance=\sepverb](c1){$+$};
		\node[input,right of=s1,node distance=.5\sephonr](output12){};
		%\node[sum,above of=output11,node distance=\sepverb](c11){$+1$};
		\node[sum,above of=output12,node distance=\sepverb](c12){$+$};
		\node[input,right of=s12,node distance=.5\sephonr](output13){};
		\node[input,above of=output13,node distance=\sepverb](output15){};
		\node[input,above of=input,node distance=\sepverb](output16){};
		\node[input,below of=input,node distance=.5\sepverb](output17){};
		% placement des flêches
		\draw[draw,->](s11)--(s10);
		\draw[draw,->](s12)--(s11);
		% première rangée de flèches verticales
		%\draw[draw,->](output11)--(c11);
		\draw[draw,->](output12)--(c12);
		% deuxième rangée de flèches verticales
		\draw[draw](input)--(s10);
		\draw[draw](input)--(output16);
		%\draw[draw](output6)--(input);
		\draw[draw,->](output16)--(c12);
		%\draw[draw,->](c11)--(c12);
		\draw[draw,->](output13)--(s12);
		\draw[draw](output13)--(output15);
		\draw[draw](c12)--(output15);
		\draw[draw,<-](output17)--(s10.west);
	\end{tikzpicture}

\item


L'état des registres ainsi que les différentes sorties :

Registre 1 :

\begin{table}[h]
		\centering
		\begin{tabular}{|c|c|c|c|c|}
			\hline
			\rowcolor[HTML]{EFEFEF} \hline
			\cellcolor[HTML]{EFEFEF}\textbf{Temps}  & \cellcolor[HTML]{EFEFEF}\textbf{sortie} & \cellcolor[HTML]{EFEFEF}\textbf{$ S_{0} $} & \cellcolor[HTML]{EFEFEF}\textbf{$ S_{1} $} & \cellcolor[HTML]{EFEFEF}\textbf{$ S_{2} $} \\ \hline
			0 &   & 1 & 1 & 1 \\ \hline
			1 & 1 & 1 & 1 & 1 \\ \hline
		
		\end{tabular}
		%\caption{My caption}
		%\label{my-label}
	\end{table}
	\item
L'état des registres ainsi que les différentes sorties :
Registre 2 :

\begin{table}[h]
		\centering
		\begin{tabular}{|c|c|c|c|c|}
			\hline
			\rowcolor[HTML]{EFEFEF} \hline
			\cellcolor[HTML]{EFEFEF}\textbf{Temps}  & \cellcolor[HTML]{EFEFEF}\textbf{sortie} & \cellcolor[HTML]{EFEFEF}\textbf{$ S_{0} $} & \cellcolor[HTML]{EFEFEF}\textbf{$ S_{1} $} & \cellcolor[HTML]{EFEFEF}\textbf{$ S_{2} $} \\ \hline
			0 &   & 1 & 1 & 1 \\ \hline
			1 & 1 & 1 & 1 & 0 \\ \hline
			2 & 1 & 1 & 0 & 1 \\ \hline
			3 & 1 & 0 & 1 & 0 \\ \hline
			4 & 0 & 1 & 0 & 0 \\ \hline
			5 & 1 & 0 & 0 & 1 \\ \hline
			6 & 0 & 0 & 1 & 1 \\ \hline
			7 & 0 & 1 & 1 & 1 \\ \hline
		\end{tabular}
		%\caption{My caption}
		%\label{my-label}
	\end{table}

%QUESTION 2

\item
Les relations de récurrence sont les suivantes :

Registre 1 : $U_{n+3}=U_n+U_{n+1}$

Registre 2 : $U_{n+3}=U_n+U_{n+2}$
\end{enumerate}
\bigskip
\item

	\begin{enumerate}(a)
		\item
		Registre tournant dans le sens des aiguilles d'une montre à 7 cases avec re-bouclage sur les cases 1 et 4 avec une initialisation à 1 pour toutes les cases.
		
\begin{tikzpicture}[auto, node distance=1.5cm,>=latex,line width=.5pt]
			% définitions de longueurs
		
		\setlength{\sephonr}{2cm}
		
		\setlength{\sepvera}{2cm}
		
		\setlength{\sepverb}{2cm}
		% placement des blocks
		\node[input,name=input]{};
		\node[block,right of=input,node distance=2cm](s0){$1$};
		\node[block,right of=s0,node distance=\sephonr](s1){$1$};
		\node[block,right of=s1,node distance=\sephonr](s2){$1$};
		\node[block,right of=s2,node distance=\sephonr](s3){$1$};
		\node[block,right of=s3,node distance=\sephonr](s4){$1$};
		\node[block,right of=s4,node distance=\sephonr](s5){$1$};
		\node[block,right of=s5,node distance=\sephonr](s6){$1$};
		\node[input,right of=s0,node distance=.5\sephonr](output1){};
	
		\node[sum,above of=output2,node distance=\sepverb](c1){$+$};
		%\node[sum,above of=output1,node distance=\sepverb](c2){$+1$};
		
		\node[input,right of=s1,node distance=.5\sephonr](output2){};
		\node[input,right of=s3,node distance=.5\sephonr](output3){};
				
		
		%\node[sum,above of=output1,node distance=\sepverb](c12){$+$};
		\node[input,right of=s6,node distance=.5\sephonr](output3){};
		\node[input,above of=output3,node distance=\sepverb](output5){};
		\node[input,above of=output1,node distance=\sepverb](output6){};
		\node[input,below of=input,node distance=.5\sepverb](output7){};
		% placement des flêches
		\draw[draw,->](s1)--(s0);
		\draw[draw,->](s2)--(s1);
		\draw[draw,->](s3)--(s2);
		\draw[draw,->](s4)--(s3);
		\draw[draw,->](s5)--(s4);
		\draw[draw,->](s6)--(s5);		
		% première rangée de flèches verticales
		\draw[draw,->](output12)--(c1);
		%\draw[draw,->](output1)--(output2);
		%\draw[draw,->](output1)--(output15);
		% deuxième rangée de flèches verticales
		%\draw[draw](input)--(s0);
		\draw[draw](c1)--(output5);
		\draw[draw](output1)--(output6);
		%\draw[draw](output6)--(input);%trait vertical de gauche
		\draw[draw,->](output6)--(c1);
		%\draw[draw,->](c12)--(c1);
		\draw[draw,->](output3)--(s6);
		\draw[draw](output3)--(output5);
		%\draw[draw](c1)--(output1);
		\draw[draw,<-](output7)--(s0.west);
	\end{tikzpicture}		
		
		
		\item
		Le polynôme caractéristique du registre :
		\textbf{$X^7+X^4+X$}
		\item
		La relation de récurrence est la suivante :
		$U_{n+7}=U_{n+1}+U_{n+4}$
		
		\item
		L'état des registres ainsi que les différentes sorties :

		\begin{table}[h]
		\centering
		\begin{tabular}{|c|c|c|c|c|c|c|c|c|}
			\hline
			\rowcolor[HTML]{EFEFEF} \hline
			\cellcolor[HTML]{EFEFEF}\textbf{Temps}  & \cellcolor[HTML]{EFEFEF}\textbf{sortie} & \cellcolor[HTML]{EFEFEF}\textbf{$ S_{0} $} & \cellcolor[HTML]{EFEFEF}\textbf{$ S_{1} $} & \cellcolor[HTML]{EFEFEF}\textbf{$ S_{2} $} & \textbf{$ S_{3} $} & \textbf{$ S_{4} $} & \textbf{$ S_{5} $} & \textbf{$ S_{6} $}\\ \hline
			0 &   & 1 & 1 & 1  &  1 & 1 & 1 & 1 \\ \hline
			1 & 1 & 1 & 1 & 1 & 1 & 1 & 1 & 0 \\ \hline
			2 & 1 & 1 & 1 & 1 & 1 & 1 & 0 & 0 \\ \hline
			3 & 1 & 1 & 1 & 1 & 1 & 0 & 0 & 0\\ \hline
			4 & 1 & 1 & 1 & 1 &  0 & 0 & 0 & 1 \\ \hline
			5 & 1 & 1 & 1 & 0& 0 & 0 & 1 & 1\\ \hline
			6 & 1 & 1 & 0 & 0&  0 & 1 & 1 &1 \\ \hline
			7 & 1 & 0 & 0 & 0&  1 & 1 & 1 &1 \\ \hline
			8 & 0 & 0 & 0 & 1&  1 & 1 & 1 &1 \\ \hline
			9 & 0 & 0 & 1 & 1&  1 & 1 & 1 &1 \\ \hline
			10 &0 & 1 & 1 & 1&  1 & 1 & 1 &0 \\ \hline
	\end{tabular}
		%\caption{My caption}
		%\label{my-label}
	\end{table}

		
	\end{enumerate}
\end{enumerate}
%%%%%%%%%%%%%%%%%%%%%%%%%%%%%%%%%%%%%%%%%%%%%%%%%%%%%%%%%%%%%%%%%%%%%%%%%%%%%
%%%%%%%%%%%%%%%%%%%%%%%%%%%%%%%%%%%%%%%%%%%%%%%%%%%%%%%%%%%%%%%%%%%%%%%%%%%%%
%%%%%%%%%%%%%%%%%%%%%%%%%%%%%%%%%%%%%%%%%%%%%%%%%%%%%%%%%%%%%%%%%%%%%%%%%%%%%



 \end{document}
