\documentclass[a4paper,10pt]{article}

\usepackage[latin1]{inputenc}
\usepackage[T1]{fontenc}
\usepackage[francais]{babel}
\usepackage{amsthm}
\usepackage{amsmath}
\usepackage{graphics}
\usepackage{amssymb}
\usepackage[top=2cm,bottom=2cm,left=2cm,right=2cm]{geometry}
%\usepackage{mathsfs}

%\title{Devoir 1 de Math�matiques}
%\author{\textsc{RIMBON} Loy}
%\date{\today}

%!!!!!!!!!!!!!!!!!!!!!!!!!!!!               !!!!!!!!!!!!!!!!!!!!!!!!!!!!
%!!!!!!!!!!!!!!!!!!!!!!!!!!!!   Exercices   !!!!!!!!!!!!!!!!!!!!!!!!!!!!
%!!!!!!!!!!!!!!!!!!!!!!!!!!!!               !!!!!!!!!!!!!!!!!!!!!!!!!!!!

\newcounter{exercice}
\setcounter{exercice}{3}
\newcommand{\exercice}{%
	\refstepcounter{exercice}%
	\bigskip
	\bigskip
	\noindent\textbf{Exercice \theexercice.}~%
}

%!!!!!!!!!!!!!!!!!!!                                !!!!!!!!!!!!!!!!!!!
%!!!!!!!!!!!!!!!!!!!   Vos commandes personnelles   !!!!!!!!!!!!!!!!!!!
%!!!!!!!!!!!!!!!!!!!                                !!!!!!!!!!!!!!!!!!!

\newcommand{\stagiaire}{Loy RIMBON}
\newcommand{\datedevoir}{27 septembre 2016}
\newcommand{\titre}{Devoir \no 3bis}

\newcommand{\R}{\mathbb{R}}
\newcommand{\C}{\mathbb{C}}
\newcommand{\N}{\mathbb{N}}
\newcommand{\Z}{\mathbb{Z}}
\newcommand{\Q}{\mathbb{Q}}
\newcommand{\LCI}{$E\times E\longrightarrow E$ ~}

\renewcommand{\leq}{\leqslant}
\renewcommand{\geq}{\geqslant}


\begin{document}
%\maketitle

\noindent\stagiaire\hfill\datedevoir

\bigskip
\bigskip
\begin{center}
{\large\bfseries\titre}
\end{center}
\bigskip
\bigskip

%%%%%%%%%%%%%%%%%%%%%%%%%%%%%%%%%%%%%%%%%%%%%%%%%%%%%%%%%%%%%%%%%%%%%%%%%%%%%
%%%%%%%%%%%%%%%%%%%%%%%%%%%%%%%%%%%%%%%%%%%%%%%%%%%%%%%%%%%%%%%%%%%%%%%%%%%%%
%%%%%%%%%%%%%%%%%%%%%%%%%%%%%%%%%%%%%%%%%%%%%%%%%%%%%%%%%%%%%%%%%%%%%%%%%%%%%

%%%%%%%%%%%%%%%%%%%%%%%%%%%%%%%%%%%%%%%%%%%%%%%%%%%%%%%%%%%%%%%%%%%%%%%%%%%%%
%%%%%%%%%%%%%%%%%%%%%%%%%%%%%%%%%%%%%%%%%%%%%%%%%%%%%%%%%%%%%%%%%%%%%%%%%%%%%
%%%%%%%%%%%%%%%%%%%%%%%%%%%%%%%%%%%%%%%%%%%%%%%%%%%%%%%%%%%%%%%%%%%%%%%%%%%%%
\exercice 

Soit $A$ un anneau. un �l�ment $a$ de $A$ est nilpotent s'il existe un entier naturel $n$ non nul tel que $a^n=0$

\begin{enumerate}

%QUESTION 1
\item
	On cherche � montrer que si $a$ est nilpotent $(1-a)$ est inversible.
	
\begin{align}
	(1^n-a^n) &= (1-a)(\displaystyle\sum_{k=0}^{n-1}a^k)	\notag\\
	1 &=(1-a)\frac{1-a^n}{1-a} \notag
\end{align}

On peut donc en d�duire que $(1-a)$ est inversible, et $\frac{1-a^n}{1-a} $ est son inverse.
\bigskip

%QUESTION 2
\item
$a$ et $b$ deux �l�ments nilpotents et permutable. $\exists n$ et $m$ tel que : $a^n=0$ et $b^n=0$. 

$(ab)^n=a^nb^b=0 $ et $ (ab)^m=a^mb^m=0$


\begin{align}
	(a+b)^{m+n} &= \displaystyle\sum_{k=0}^{m+n} C_{m+n}^ka^{m+n-k}b^{k} \notag\\
	&=a^n(\displaystyle\sum_{k=0}^{m-1}C_{m}^ka^{m-k}b^{k} )+b^m(\displaystyle\sum_{k=m}^{m+n}C_{n}^ka^{m+n-k}b^{k-m} )\notag\\
	&=0 \notag
\end{align}

Donc si $a$ et $b$ sont nilpotents, $ab$ et $(a+b)$ sont nilpotents.
\bigskip

%QUESTION 3
\item
Soit $B$ l'ensemble des �l�ments nilpotents. $B \subset A$ et $\forall  b \in B, \exists n,n\neq 0$ tel que $b^n=0$.

L'ensemble $B$ n'est pas vide car l'�l�ment ${0} \in B, 0^1=0$.

Soit $b \in B, a \in A$, le produit $ab \in B $ car $(ab)^n=0$.


Soit $a$ et $b$ nilpotent, $a$ et $b \in B$. D'apr�s le (2) on sait que $(a+b) $est nilpotent. Donc $(a+b) \in A$.


On peut donc en d�duire que l'ensemble $B$ des �l�ments nilpotents de $A$ forme un id�al de $A$.

\end{enumerate}

%%%%%%%%%%%%%%%%%%%%%%%%%%%%%%%%%%%%%%%%%%%%%%%%%%%%%%%%%%%%%%%%%%%%%%%%%%%%%
%%%%%%%%%%%%%%%%%%%%%%%%%%%%%%%%%%%%%%%%%%%%%%%%%%%%%%%%%%%%%%%%%%%%%%%%%%%%%
%%%%%%%%%%%%%%%%%%%%%%%%%%%%%%%%%%%%%%%%%%%%%%%%%%%%%%%%%%%%%%%%%%%%%%%%%%%%%
\exercice 
Soit $A$ un anneau de Boole tel que $\forall x \in A, x^2=x$.

\begin{enumerate}

%QUESTION 1
\item
\begin{align}
	x^2 &= x \notag \\
	x^2+x+x+x &= x+x+x+x \notag \\
	x^2+x+x+x^2 &=x+x+x+x \notag \\
	(x+x)^2&=x+x+x+x \notag \\
	x+x &= x+x+x+x \notag \\
	x+x &=0 \notag
\end{align}

Donc $x=-x$, quelque soit $x \in A $, $x^{-1}=x$.
%QUESTION 2
\item
Soit $(x,y) \in A$
\begin{align}
	x+y &= (x+y)^2  \notag \\
	x+y&= xx+xy+yx+yy \notag \\
	x+y&= x^2 + xy +yx + y^2 \notag \\
	x+y &= x + xy +yx + y \notag \\
	 x+y &= x+y+xy+yx \notag 
\end{align}
Et donc :% $	xy +yx = 0 $
\begin{align}
	xy +yx &= 0 \notag \\
	xy = -yx \notag 
\end{align}
De plus, d'apr�s la question pr�c�dente : $x+x=0$ et $x=-x$.

Donc $-yx=yx \implies xy=yx$, l'anneau $A$ est commutatif.
\bigskip

%QUESTION 3
\item
Soit $(x,y) \in A$. On sait d'apr�s la question 2 que $x+x=0 $,
\begin{align}
	xy(x+y) &= xyx + xyy \notag \\
	&= xxy + xy^2 \notag \\
	&= x^2y +xy^2 \notag \\
	&=xy+xy \notag \\
	&=y(x+x) \notag \\
	xy(x+y)&=0 \notag
\end{align}
%QUESTION 4
\item
Soit $(x,y) \in A$ tel que $x \neq 0$ et $y \neq 0$. L'anneau $A$ poss�de donc au moins 3 �l�ments, $0 \in A$.

D'apr�s la question 3 on peut �crire l'�galit� suivante :
\begin{align}
	xy(x+y) &= 0 \notag 
\end{align}
Or on sait que $(x+y) \neq 0$, car $x\neq y$ et $y$ n'est pas l'oppos� de $x$ car d'apr�s la question 1, un �l�ment de $A$ est son propre inverse.
Donc :
\begin{align}
	xy(x+y) &= \implies xy=0 \notag 
\end{align}

On peut donc en d�duire que si $A$ � plus de 2 �l�ments, $A$ n'est pas int�gre.
\end{enumerate}
%%%%%%%%%%%%%%%%%%%%%%%%%%%%%%%%%%%%%%%%%%%%%%%%%%%%%%%%%%%%%%%%%%%%%%%%%%%%%
%%%%%%%%%%%%%%%%%%%%%%%%%%%%%%%%%%%%%%%%%%%%%%%%%%%%%%%%%%%%%%%%%%%%%%%%%%%%%
%%%%%%%%%%%%%%%%%%%%%%%%%%%%%%%%%%%%%%%%%%%%%%%%%%%%%%%%%%%%%%%%%%%%%%%%%%%%%



 \end{document}
