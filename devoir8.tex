	\documentclass[a4paper,10pt]{article}
	
	\usepackage[utf8]{inputenc}
	\usepackage[T1]{fontenc}
	\usepackage[francais]{babel}
	\usepackage{amsthm}
	\usepackage{amsmath}
	\usepackage{graphics}
	\usepackage{amssymb}
	\usepackage{color}
	\usepackage[table,xcdraw]{xcolor}
	\usepackage{tikz}
	
	\usepackage{xlop}
	\usepackage{ifthen}
	\usepackage[top=2cm,bottom=2cm,left=2cm,right=2cm]{geometry}

	%\usepackage{mathsfs}
	
	%\title{Devoir 1 de Mathématiques}
	%\author{\textsc{RIMBON} Loy}
	%\date{\today}
	
	%!!!!!!!!!!!!!!!!!!!!!!!!!!!!               !!!!!!!!!!!!!!!!!!!!!!!!!!!!
	%!!!!!!!!!!!!!!!!!!!!!!!!!!!!   Exercices   !!!!!!!!!!!!!!!!!!!!!!!!!!!!
	%!!!!!!!!!!!!!!!!!!!!!!!!!!!!               !!!!!!!!!!!!!!!!!!!!!!!!!!!!
	
	\newcounter{exercice}
	\setcounter{exercice}{0}
	\newcommand{\exercice}{%
		\refstepcounter{exercice}%
		\bigskip
		\bigskip
		\noindent\textbf{Exercice \theexercice.}~%
	}
	%%%%%%%%%%%%%%%%%%%%%%%%%%%%%%%%%%%%%%%%%%%%%%%%%%%%%%%
	% styles TikZ pour les blocks
	\tikzstyle{block}=[draw,rectangle,fill=white,inner sep=0mm,minimum size=12mm]
	\tikzstyle{emptyblock}=[rectangle,fill=white,inner sep=0mm,minimum size=12mm]
	\tikzstyle{sumg}=[draw,circle,node distance=1cm,inner sep=0mm,minimum size=10mm]
	\tikzstyle{sum}=[draw,circle,node distance=1cm,inner sep=0mm,minimum size=6mm]
	\tikzstyle{input}=[coordinate]
	%%%%%%%%%%%%%%%%%%%%%%%%%%%%%%%%%%%%%%%%%%%%%%%%%%%%%%%
	%!!!!!!!!!!!!!!!!!!!                                !!!!!!!!!!!!!!!!!!!
	%!!!!!!!!!!!!!!!!!!!   Vos commandes personnelles   !!!!!!!!!!!!!!!!!!!
	%!!!!!!!!!!!!!!!!!!!                                !!!!!!!!!!!!!!!!!!!
	
	\newcommand{\stagiaire}{Loy RIMBON}
	\newcommand{\datedevoir}{3 novembre 2016}
	\newcommand{\titre}{Devoir \no 8}
	
	\newcommand{\R}{\mathbb{R}}
	\newcommand{\C}{\mathbb{C}}
	\newcommand{\N}{\mathbb{N}}
	\newcommand{\Z}{\mathbb{Z}}
	\newcommand{\Q}{\mathbb{Q}}
	\newcommand{\LCI}{$E\times E\longrightarrow E$ ~}
	
	\renewcommand{\leq}{\leqslant}
	\renewcommand{\geq}{\geqslant}
	
	\newcounter{arret}%
\setcounter{arret}{0}%
\newcommand{\PGCD}[2]%
{%
\opcopy{#1}{a}%
\opcopy{#2}{b}%
\opcopy{#1}{A}%
\opcopy{#2}{B}%
\opgcd{A}{B}{PGCD}%
\noindent%
%Calculons par l'algorithme d'\textsc{Euclide} le PGCD des nombres $ \opprint{A} $ et $ \opprint{B} $.\\%
%\begin{align}


\whiledo{\equal{\thearret}{0}}%
{\opidiv*{a}{b}{q}{r}%
$ \opprint{a} = \opprint{b} \times \opprint{q} + \opprint{r} $\\
%\opprint{a} $= \opprint{b} \times \opprint{q} + \opprint{r} \notag   \\
\opcmp{r}{0}%
\ifopeq%
        \refstepcounter{arret}%
\fi%
%\end{align}
\opcopy{b}{a}%
\opcopy{r}{b}}%
%Le PGCD des nombres $ \opprint{A} $ et $ \opprint{B} $ est le dernier reste non nul du procédé,
%c'est-à-dire $ \opprint{PGCD} $.%
}%
	
	\begin{document}
	%\maketitle
	
	\noindent\stagiaire\hfill\datedevoir
	
	\bigskip
	\bigskip
	\begin{center}
	{\large\bfseries\titre}
	\end{center}
	\bigskip
	\bigskip
	
	%%%%%%%%%%%%%%%%%%%%%%%%%%%%%%%%%%%%%%%%%%%%%%%%%%%%%%%%%%%%%%%%%%%%%%%%%%%%%
%%%%%%%%%%%%%%%%%%%%%%%%%%%%%%%%%%%%%%%%%%%%%%%%%%%%%%%%%%%%%%%%%%%%%%%%%%%%%
%%%%%%%%%%%%%%%%%%%%%%%%%%%%%%%%%%%%%%%%%%%%%%%%%%%%%%%%%%%%%%%%%%%%%%%%%%%%%
	\exercice Chiffrement affine.
	
	\begin{enumerate}
	\item
	
	\item
NTJMPUMGXPQTSTGQPGTXPNCHUMTPUTGFSFTGTHNNGXNCHUMWXOOTRTUMH
PYCTGKTJQTJCHFOOXUJQHGZTUMXPOTJXOTFOQTOHRXUMHZUTWFTGTOPFMNT
JMPUATMFMSHODPFRXPJJTQTGHBXUJ


\begin{tabular}{|c|c|c|}
	\hline 
	LETTRE & Nombre d'apparition & 19,05 \\ 
	\hline 
	T & 24 & 19,05 \\ 
	\hline 
	M & 11 & 8,73 \\ 
	\hline 
	P & 11 & 8,73 \\ 
	\hline 
	U& 11 & 8,73 \\ 
	\hline 
	G	&10	&7,93\\ 
	\hline 
	X	&9	& 7,14\\ 
	\hline 
	H	&8	& 6,35\\ 
	\hline 
	J & 8 & 6,35 \\ 
	\hline 
	N & 6 & 4,76	 \\ 
	\hline 
	Q	&6	& 4,76\\ 
	\hline 
	F	&6	& 4,76\\ 
	\hline 
	C	&4	& 3,17\\ 
	\hline 
	S	&3	& 2,38\\ 
	\hline 
	R	&3	& 2,38\\ 
	\hline 
	W	&2	& 1,58\\ 
	\hline 
	Y	&1	& 0,79\\ 
	\hline 
	K	&1	& 0,79\\ 
	\hline 
	A	&1	& 0,79\\ 
	\hline 
	B	& 1	&0,79\\ 
	\hline 
	
	\end{tabular} 	

La lettre T, a une fréquence d'apparition de près de 19\%. On peut logiquement déduire que T représente la lettre e.

Or un chiffrement affine et du type : lettre chiffrée $= a\times i + b \pmod{26}$.
On à donc une première équation : $19=4a+b \pmod{26}$. 19 et 4 étant respectivement les rangs de T et de e.

D'après les fréquences calculées, il y'a 7 candidats crédibles pour la lettre a : M, P, U, G,X,H et J.
Le rang de a étant 0, la deuxième équation sera de la forme : rang de la lettre chiffrée = b $\pmod{26}$ et $a=\frac{19-b}{4}$. Ce qui nous donne les différentes possibilités de couple :

\begin{align}
 M=12 \implies a&=\frac{19-12}{4}=\frac{7}{4}\pmod{26} \\
 P=15 \implies a&=\frac{19-15}{4}=1\pmod{26}\\
 U=20 \implies a&=\frac{19-20}{4}=\frac{-1}{4}\pmod{26}\\
 G=6 \implies a&=\frac{19-6}{4}=\frac{13}{4}\pmod{26}\\
 X=23 \implies a&=\frac{19-23}{4}=25\pmod{26}\\
 H=7 \implies a&=\frac{19-7}{4}=3\pmod{26}\\
 J=9 \implies a&=\frac{19-9}{4}=\frac{10}{4}\pmod{26}
\end{align}

Après calcul de a, il n'y a plus que 3 possibilités issues des équations (2),(5) et (6), $a \Z$.
\begin{align}
a=1,b=15 &\implies (b,p) : i+b \pmod{26} \notag \\
a=25,b=23 &\implies (z,x):25i+23 \pmod{26} \notag \\
a=3,b=7 &\implies (d,h) : 3i+7 \pmod{26} \notag 
\end{align}

Des trois équations restantes, il n y'a que celle du couple (d,h) qui appliquée à la lettre e renvoi la lettre T.

On applique cette équations à toute les lettres de l'alphabet :

\begin{tabular}{|c|c|c|c|}
	\hline 
	Lettre clair & rang & xi & Lettre chiffrée \\ 
	\hline
	a	& 0	& 7	& H\\ 
	\hline
	b	& 1	& 10	& K\\ 
	\hline
	c	& 2	& 13	& N\\ 
	\hline
	d	& 3	& 16	& Q\\ 
	\hline
	e	& 4	& 19	& T\\ 
	\hline
	f	& 5	& 22	& W\\ 
	\hline
	g	& 6	& 25	& Z\\ 
	\hline
	h	& 7	& 	2	& C\\ 
	\hline
	i	& 8	& 	5	& F\\ 
	\hline
	j	& 9	& 	8	& I\\ 
	\hline
	k	& 10	& 	11& 	L\\ 
	\hline
	l	& 11	& 	14& 	O\\ 
	\hline
	m	& 12	& 	17	& R\\ 
	\hline
	n	& 13	& 20	& U\\ 
	\hline
	o	& 14	& 	23	& X\\ 
	\hline
	p	& 15	& 	0	& A\\ 
	\hline
	q	& 16	& 	3	& D\\ 
	\hline
	r	& 17	& 	6	& G\\ 
	\hline
	s	& 18	& 	9	& J\\ 
	\hline
	t	& 19	& 	12	& M\\ 
	\hline
	u	& 20	& 15	& P\\ 
	\hline
	v	& 21	& 	18	& S\\ 
	\hline
	w	& 22	& 	21	& V\\ 
	\hline
	x	& 23	& 	24	& Y\\ 
	\hline
	y	& 24	& 	1	& B\\ 
	\hline
	z	& 25	& 	4	& E\\ 
	\hline
	\end{tabular} 	
	
En appliquant cette alphabet au texte chiffrée on retrouve le texte suivant :

c'est un trou de verdure ou chante une riviere accrochant follement aux herbes des haillons d'argent ou le soleil de la montagne fiere luit c est un petit val qui mousse de rayons.
         

	\end{enumerate}		

	%%%%%%%%%%%%%%%%%%%%%%%%%%%%%%%%%%%%%%%%%%%%%%%%%%%%%%%%%%%%%%%%%%%%%%%%%%%%%
%%%%%%%%%%%%%%%%%%%%%%%%%%%%%%%%%%%%%%%%%%%%%%%%%%%%%%%%%%%%%%%%%%%%%%%%%%%%%
%%%%%%%%%%%%%%%%%%%%%%%%%%%%%%%%%%%%%%%%%%%%%%%%%%%%%%%%%%%%%%%%%%%%%%%%%%%%%
	\exercice 
	
	On considère dans $F_2[X]$ le polynôme $P(X)=X^6+X^5+1$ et $A$ l'anneau $F-2[X]/(P(X))$.
	\begin{enumerate}
	\item
	Soit $\alpha$ la classe de $X$ dans $A$.Montrer qu $\alpha$ est une unité.
	\item
	\item
	\item
	\item
	\end{enumerate}	
%%%%%%%%%%%%%%%%%%%%%%%%%%%%%%%%%%%%%%%%%%%%%%%%%%%%%%%%%%%%%%%%%%%%%%%%%%%%%
%%%%%%%%%%%%%%%%%%%%%%%%%%%%%%%%%%%%%%%%%%%%%%%%%%%%%%%%%%%%%%%%%%%%%%%%%%%%%
%%%%%%%%%%%%%%%%%%%%%%%%%%%%%%%%%%%%%%%%%%%%%%%%%%%%%%%%%%%%%%%%%%%%%%%%%%%%%
	\exercice
	\begin{enumerate}
	\item
	Décomposer 341 en produit de facteurs premier :
	
	$341=11 \times 31$
	\item
	Montrer d'après le petit théorème de Fermat que $341$ n'est pas  premier.
	On cherche donc à calculer $a^{340}$, pour $a \in Z$.	
	On choisit $a=2$.
	
	\begin{align}
		2^{340} &\equiv (2^10)^3 \times 2^4 \pmod{341}\notag \\
			  &\equiv 1024^3 \times 16 \pmod{341}\notag \\
			  &\equiv 1^3 \times 16 \pmod{341}\notag \\
			  &\equiv 16 \pmod{341}\notag
	\end{align}
	
	Ainsi, $2^{340}$ n'est pas congru à $1 \pmod{341}$, $341$ n'est pas premier.
	\end{enumerate}
		
	 \end{document}
		