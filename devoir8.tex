	\documentclass[a4paper,10pt]{article}
	
	\usepackage[utf8]{inputenc}
	\usepackage[T1]{fontenc}
	\usepackage[francais]{babel}
	\usepackage{amsthm}
	\usepackage{amsmath}
	\usepackage{graphics}
	\usepackage{amssymb}
	\usepackage{color}
	\usepackage[table,xcdraw]{xcolor}
	\usepackage{tikz}
	
	\usepackage{xlop}
	\usepackage{ifthen}
	\usepackage[top=2cm,bottom=2cm,left=2cm,right=2cm]{geometry}

	%\usepackage{mathsfs}
	
	%\title{Devoir 1 de Mathématiques}
	%\author{\textsc{RIMBON} Loy}
	%\date{\today}
	
	%!!!!!!!!!!!!!!!!!!!!!!!!!!!!               !!!!!!!!!!!!!!!!!!!!!!!!!!!!
	%!!!!!!!!!!!!!!!!!!!!!!!!!!!!   Exercices   !!!!!!!!!!!!!!!!!!!!!!!!!!!!
	%!!!!!!!!!!!!!!!!!!!!!!!!!!!!               !!!!!!!!!!!!!!!!!!!!!!!!!!!!
	
	\newcounter{exercice}
	\setcounter{exercice}{0}
	\newcommand{\exercice}{%
		\refstepcounter{exercice}%
		\bigskip
		\bigskip
		\noindent\textbf{Exercice \theexercice.}~%
	}
	%%%%%%%%%%%%%%%%%%%%%%%%%%%%%%%%%%%%%%%%%%%%%%%%%%%%%%%
	% styles TikZ pour les blocks
	\tikzstyle{block}=[draw,rectangle,fill=white,inner sep=0mm,minimum size=12mm]
	\tikzstyle{emptyblock}=[rectangle,fill=white,inner sep=0mm,minimum size=12mm]
	\tikzstyle{sumg}=[draw,circle,node distance=1cm,inner sep=0mm,minimum size=10mm]
	\tikzstyle{sum}=[draw,circle,node distance=1cm,inner sep=0mm,minimum size=6mm]
	\tikzstyle{input}=[coordinate]
	%%%%%%%%%%%%%%%%%%%%%%%%%%%%%%%%%%%%%%%%%%%%%%%%%%%%%%%
	%!!!!!!!!!!!!!!!!!!!                                !!!!!!!!!!!!!!!!!!!
	%!!!!!!!!!!!!!!!!!!!   Vos commandes personnelles   !!!!!!!!!!!!!!!!!!!
	%!!!!!!!!!!!!!!!!!!!                                !!!!!!!!!!!!!!!!!!!
	
	\newcommand{\stagiaire}{Loy RIMBON}
	\newcommand{\datedevoir}{3 novembre 2016}
	\newcommand{\titre}{Devoir \no 8}
	
	\newcommand{\R}{\mathbb{R}}
	\newcommand{\C}{\mathbb{C}}
	\newcommand{\N}{\mathbb{N}}
	\newcommand{\Z}{\mathbb{Z}}
	\newcommand{\Q}{\mathbb{Q}}
	\newcommand{\LCI}{$E\times E\longrightarrow E$ ~}
	
	\renewcommand{\leq}{\leqslant}
	\renewcommand{\geq}{\geqslant}
	
	\newcounter{arret}%
\setcounter{arret}{0}%
\newcommand{\PGCD}[2]%
{%
\opcopy{#1}{a}%
\opcopy{#2}{b}%
\opcopy{#1}{A}%
\opcopy{#2}{B}%
\opgcd{A}{B}{PGCD}%
\noindent%
%Calculons par l'algorithme d'\textsc{Euclide} le PGCD des nombres $ \opprint{A} $ et $ \opprint{B} $.\\%
\whiledo{\equal{\thearret}{0}}%
{\opidiv*{a}{b}{q}{r}%
$ \opprint{a} = \opprint{b} \times \opprint{q} + \opprint{r} $\\%
\opcmp{r}{0}%
\ifopeq%
        \refstepcounter{arret}%
\fi%
\opcopy{b}{a}%
\opcopy{r}{b}}%
%Le PGCD des nombres $ \opprint{A} $ et $ \opprint{B} $ est le dernier reste non nul du procédé,
%c'est-à-dire $ \opprint{PGCD} $.%
}%
	
	\begin{document}
	%\maketitle
	
	\noindent\stagiaire\hfill\datedevoir
	
	\bigskip
	\bigskip
	\begin{center}
	{\large\bfseries\titre}
	\end{center}
	\bigskip
	\bigskip
	
	%%%%%%%%%%%%%%%%%%%%%%%%%%%%%%%%%%%%%%%%%%%%%%%%%%%%%%%%%%%%%%%%%%%%%%%%%%%%%
%%%%%%%%%%%%%%%%%%%%%%%%%%%%%%%%%%%%%%%%%%%%%%%%%%%%%%%%%%%%%%%%%%%%%%%%%%%%%
%%%%%%%%%%%%%%%%%%%%%%%%%%%%%%%%%%%%%%%%%%%%%%%%%%%%%%%%%%%%%%%%%%%%%%%%%%%%%
	\exercice Chiffrement affine.
	
	\begin{enumerate}
	\item
	
	\item
NTJMPUMGXPQTSTGQPGTXPNCHUMTPUTGFSFTGTHNNGXNCHUMWXOOTRTUMH
PYCTGKTJQTJCHFOOXUJQHGZTUMXPOTJXOTFOQTOHRXUMHZUTWFTGTOPFMNT
JMPUATMFMSHODPFRXPJJTQTGHBXUJ


\begin{tabular}{|c|c|c|}
	\hline 
	N & 6 & 4,76	 \\ 
	\hline 
	T & 24 & 19,05 \\ 
	\hline 
	J & 8 & 6,35 \\ 
	\hline 
	M & 11 & 8,73 \\ 
	\hline 
	P & 11 & 8,73 \\ 
	\hline 
	U& 11 & 8,73 \\ 
	\hline 
	• & • & • \\ 
	\hline 
	• & • & • \\ 
	\hline 
	• & • & • \\ 
	\hline 
	• & • & • \\ 
	\hline 
	• & • & • \\ 
	\hline 
	• & • & • \\ 
	\hline 
	• & • & • \\ 
	\hline 
	• & • & • \\ 
	\hline 
	• & • & • \\ 
	\hline 
	• & • & • \\ 
	\hline 
	• & • & • \\ 
	\hline 
	• & • & • \\ 
	\hline 
	• & • & • \\ 
	\hline 
	\end{tabular} 	

G	10	7,93
X	9	7,14
Q	6	4,76
S	3	2,38
C	4	3,17
H	8	6,35
F	6	4,76
W	2	1,58
R	3	2,38
Y	1	0,79
K	1	0,79
A	1	0,79
B	1	0,79
	
NTJMPUMGXPQTSTGQPGTXPNCHUMTPUTGFSFTGTHNNGXNCHUMWXOOTRTUMH
 cestuntroudeverdxafar-a—a                       e   e          e                   e     e          e   e                                      e   e  
PYCTGKTJQTJCHFOOXUJQHGZTUMXPOTJXOTFOQTOHRXUMHZUTWFTGTOPFMNT
        e     e     e                                 e            e        e        e                        e      e   e            e
JMPUATMFMSHODPFRXPJJTQTGHBXUJ
             e                                  e   e

	
	
	
	\end{enumerate}		

	%%%%%%%%%%%%%%%%%%%%%%%%%%%%%%%%%%%%%%%%%%%%%%%%%%%%%%%%%%%%%%%%%%%%%%%%%%%%%
%%%%%%%%%%%%%%%%%%%%%%%%%%%%%%%%%%%%%%%%%%%%%%%%%%%%%%%%%%%%%%%%%%%%%%%%%%%%%
%%%%%%%%%%%%%%%%%%%%%%%%%%%%%%%%%%%%%%%%%%%%%%%%%%%%%%%%%%%%%%%%%%%%%%%%%%%%%
	\exercice Exponentiation rapide.


%%%%%%%%%%%%%%%%%%%%%%%%%%%%%%%%%%%%%%%%%%%%%%%%%%%%%%%%%%%%%%%%%%%%%%%%%%%%%
%%%%%%%%%%%%%%%%%%%%%%%%%%%%%%%%%%%%%%%%%%%%%%%%%%%%%%%%%%%%%%%%%%%%%%%%%%%%%
%%%%%%%%%%%%%%%%%%%%%%%%%%%%%%%%%%%%%%%%%%%%%%%%%%%%%%%%%%%%%%%%%%%%%%%%%%%%%
	\exercice Algorithme d'Euclide.
	
	%%%%%%%%%%%%%%%%%%%%%%%%%%%%%%%%%%%%%%%%%%%%%%%%%%%%%%%%%%%%%%%%%%%%%%%%%%%%%
%%%%%%%%%%%%%%%%%%%%%%%%%%%%%%%%%%%%%%%%%%%%%%%%%%%%%%%%%%%%%%%%%%%%%%%%%%%%%
%%%%%%%%%%%%%%%%%%%%%%%%%%%%%%%%%%%%%%%%%%%%%%%%%%%%%%%%%%%%%%%%%%%%%%%%%%%%%
	\exercice	
	 \end{document}
		