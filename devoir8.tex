 	\documentclass[a4paper,10pt]{article}
	
	\usepackage[utf8]{inputenc}
	\usepackage[T1]{fontenc}
	\usepackage[francais]{babel}
	\usepackage{amsthm}
	\usepackage{amsmath}
	\usepackage{graphics}
	\usepackage{amssymb}
	\usepackage{color}
	\usepackage[table,xcdraw]{xcolor}
	\usepackage{tikz}
	
	\usepackage{xlop}
	\usepackage{ifthen}
	\usepackage[top=2cm,bottom=2cm,left=2cm,right=2cm]{geometry}

	%\usepackage{mathsfs}
	
	%\title{Devoir 1 de Mathématiques}
	%\author{\textsc{RIMBON} Loy}
	%\date{\today}
	
	%!!!!!!!!!!!!!!!!!!!!!!!!!!!!               !!!!!!!!!!!!!!!!!!!!!!!!!!!!
	%!!!!!!!!!!!!!!!!!!!!!!!!!!!!   Exercices   !!!!!!!!!!!!!!!!!!!!!!!!!!!!
	%!!!!!!!!!!!!!!!!!!!!!!!!!!!!               !!!!!!!!!!!!!!!!!!!!!!!!!!!!
	
	\newcounter{exercice}
	\setcounter{exercice}{0}
	\newcommand{\exercice}{%
		\refstepcounter{exercice}%
		\bigskip
		\bigskip
		\noindent\textbf{Exercice \theexercice.}~%
	}
	%%%%%%%%%%%%%%%%%%%%%%%%%%%%%%%%%%%%%%%%%%%%%%%%%%%%%%%
	% styles TikZ pour les blocks
	\tikzstyle{block}=[draw,rectangle,fill=white,inner sep=0mm,minimum size=12mm]
	\tikzstyle{emptyblock}=[rectangle,fill=white,inner sep=0mm,minimum size=12mm]
	\tikzstyle{sumg}=[draw,circle,node distance=1cm,inner sep=0mm,minimum size=10mm]
	\tikzstyle{sum}=[draw,circle,node distance=1cm,inner sep=0mm,minimum size=6mm]
	\tikzstyle{input}=[coordinate]
	%%%%%%%%%%%%%%%%%%%%%%%%%%%%%%%%%%%%%%%%%%%%%%%%%%%%%%%
	%!!!!!!!!!!!!!!!!!!!                                !!!!!!!!!!!!!!!!!!!
	%!!!!!!!!!!!!!!!!!!!   Vos commandes personnelles   !!!!!!!!!!!!!!!!!!!
	%!!!!!!!!!!!!!!!!!!!                                !!!!!!!!!!!!!!!!!!!
	
	\newcommand{\stagiaire}{Loy RIMBON}
	\newcommand{\datedevoir}{3 novembre 2016}
	\newcommand{\titre}{Devoir \no 8}
	
	\newcommand{\R}{\mathbb{R}}
	\newcommand{\C}{\mathbb{C}}
	\newcommand{\N}{\mathbb{N}}
	\newcommand{\Z}{\mathbb{Z}}
	\newcommand{\Q}{\mathbb{Q}}
	\newcommand{\LCI}{$E\times E\longrightarrow E$ ~}
	
	\renewcommand{\leq}{\leqslant}
	\renewcommand{\geq}{\geqslant}
	
	\newcounter{arret}%
\setcounter{arret}{0}%
\newcommand{\PGCD}[2]%
{%
\opcopy{#1}{a}%
\opcopy{#2}{b}%
\opcopy{#1}{A}%
\opcopy{#2}{B}%
\opgcd{A}{B}{PGCD}%
\noindent%
%Calculons par l'algorithme d'\textsc{Euclide} le PGCD des nombres $ \opprint{A} $ et $ \opprint{B} $.\\%
%\begin{align}


\whiledo{\equal{\thearret}{0}}%
{\opidiv*{a}{b}{q}{r}%
$ \opprint{a} = \opprint{b} \times \opprint{q} + \opprint{r} $\\
%\opprint{a} $= \opprint{b} \times \opprint{q} + \opprint{r} \notag   \\
\opcmp{r}{0}%
\ifopeq%
        \refstepcounter{arret}%
\fi%
%\end{align}
\opcopy{b}{a}%
\opcopy{r}{b}}%
%Le PGCD des nombres $ \opprint{A} $ et $ \opprint{B} $ est le dernier reste non nul du procédé,
%c'est-à-dire $ \opprint{PGCD} $.%
}%
	
	\begin{document}
	%\maketitle
	
	\noindent\stagiaire\hfill\datedevoir
	
	\bigskip
	\bigskip
	\begin{center}
	{\large\bfseries\titre}
	\end{center}
	\bigskip
	\bigskip
	
	%%%%%%%%%%%%%%%%%%%%%%%%%%%%%%%%%%%%%%%%%%%%%%%%%%%%%%%%%%%%%%%%%%%%%%%%%%%%%
%%%%%%%%%%%%%%%%%%%%%%%%%%%%%%%%%%%%%%%%%%%%%%%%%%%%%%%%%%%%%%%%%%%%%%%%%%%%%
%%%%%%%%%%%%%%%%%%%%%%%%%%%%%%%%%%%%%%%%%%%%%%%%%%%%%%%%%%%%%%%%%%%%%%%%%%%%%
	\exercice Chiffrement affine.
	
	\begin{enumerate}
	\item
	L'application qui à $i$ associe $a\times i+b\pmod{26}$ est une application forcément bijective sinon le décodage est impossible. De plus, $a$ est inversible. Un élément $a$ inversible de $\Z/26\Z$ est tel que : $a\wedge 26=1$.
	
Le nombre d'élément inversible dans $\Z/26\Z$ est $\varphi(26)$.

$\varphi(26)=26\times (1-\frac{1}{2})(1-\frac{1}{13})=12$.

Il y' a donc 12 éléments inversibles : 1, 3, 5, 7, 9, 11, 15, 17, 19, 21, 23, 25.
A chaque élément on peut associer 26 lettres différentes pour former un couple (a,b). Il y' a donc $12\times 26$ c'est à dire 312 applications possible.
	\item
NTJMPUMGXPQTSTGQPGTXPNCHUMTPUTGFSFTGTHNNGXNCHUMWXOOTRTUMH
PYCTGKTJQTJCHFOOXUJQHGZTUMXPOTJXOTFOQTOHRXUMHZUTWFTGTOPFMNT
JMPUATMFMSHODPFRXPJJTQTGHBXUJ

En faisant une analyse fréquentielle, on obtient le resultat suivant :

\begin{center}

	\begin{tabular}{|c|c|c|}
	\hline 
	LETTRE & Nombre d'apparition & Fréquence en \% \\ 
	\hline 
	T & 24 & 19,05 \\ 
	\hline 
	M & 11 & 8,73 \\ 
	\hline 
	P & 11 & 8,73 \\ 
	\hline 
	U& 11 & 8,73 \\ 
	\hline 
	G	&10	&7,93\\ 
	\hline 
	X	&9	& 7,14\\ 
	\hline 
	H	&8	& 6,35\\ 
	\hline 
	J & 8 & 6,35 \\ 
	\hline 
	N & 6 & 4,76	 \\ 
	\hline 
	Q	&6	& 4,76\\ 
	\hline 
	F	&6	& 4,76\\ 
	\hline 
	C	&4	& 3,17\\ 
	\hline 
	S	&3	& 2,38\\ 
	\hline 
	R	&3	& 2,38\\ 
	\hline 
	W	&2	& 1,58\\ 
	\hline 
	Y	&1	& 0,79\\ 
	\hline 
	K	&1	& 0,79\\ 
	\hline 
	A	&1	& 0,79\\ 
	\hline 
	B	& 1	&0,79\\ 
	\hline 
	
	\end{tabular} 	
\end{center}
La lettre T, a une fréquence d'apparition de près de 19\%. On peut logiquement déduire que T représente la lettre e.

Or, un chiffrement affine et du type : lettre chiffrée $= a\times i + b \pmod{26}$.

On obtient donc une première équation : $19=4a+b \pmod{26}$. 19 et 4 étant respectivement les rangs de T et de e. De plus, l'hypothèse est crédible car 19 est l'un des éléments inversible de $\Z/26\Z$.

D'après les fréquences calculées, il y'a 7 candidats crédibles pour la lettre a : M, P, U, G,X,H et J. On peut déjà éliminer les lettres(M,U,G) dont le rang ne correspond pas à un élément inversible.
Le rang de a étant 0, la deuxième équation sera de la forme : rang de la lettre chiffrée = b $\pmod{26}$ et $a=\frac{19-b}{4}$. On calcul, les différentes valeurs possibles pour $a$.

\begin{align}
 P=15 \implies a&=\frac{19-15}{4}=1\pmod{26}\\
 X=23 \implies a&=\frac{19-23}{4}=25\pmod{26}\\
 H=7 \implies a&=\frac{19-7}{4}=3\pmod{26}\\
 J=9 \implies a&=\frac{19-9}{4}=\frac{10}{4}\pmod{26}
\end{align}

Après calcul de a, il n'y a plus que 3 possibilités issues des équations (1),(2) et (3).
\begin{align}
a=1,b=15 &\implies (b,p) : i+b \pmod{26} \notag \\
a=25,b=23 &\implies (z,x):25i+23 \pmod{26} \notag \\
a=3,b=7 &\implies (d,h) : 3i+7 \pmod{26} \notag 
\end{align}

Des trois équations restantes, il n y'a que celle du couple (d,h) qui appliquée à la lettre e renvoi la lettre T.

On applique cette équations à toute les lettres de l'alphabet :

\begin{center}

\begin{tabular}{|c|c|c|c|}
	\hline 
	Lettre clair & rang & $xi=3i+7\pmod{26}$ & Lettre chiffrée \\ 
	\hline
	a	& 0	& 7	& H\\ 
	\hline
	b	& 1	& 10	& K\\ 
	\hline
	c	& 2	& 13	& N\\ 
	\hline
	d	& 3	& 16	& Q\\ 
	\hline
	e	& 4	& 19	& T\\ 
	\hline
	f	& 5	& 22	& W\\ 
	\hline
	g	& 6	& 25	& Z\\ 
	\hline
	h	& 7	& 	2	& C\\ 
	\hline
	i	& 8	& 	5	& F\\ 
	\hline
	j	& 9	& 	8	& I\\ 
	\hline
	k	& 10	& 	11& 	L\\ 
	\hline
	l	& 11	& 	14& 	O\\ 
	\hline
	m	& 12	& 	17	& R\\ 
	\hline
	n	& 13	& 20	& U\\ 
	\hline
	o	& 14	& 	23	& X\\ 
	\hline
	p	& 15	& 	0	& A\\ 
	\hline
	q	& 16	& 	3	& D\\ 
	\hline
	r	& 17	& 	6	& G\\ 
	\hline
	s	& 18	& 	9	& J\\ 
	\hline
	t	& 19	& 	12	& M\\ 
	\hline
	u	& 20	& 15	& P\\ 
	\hline
	v	& 21	& 	18	& S\\ 
	\hline
	w	& 22	& 	21	& V\\ 
	\hline
	x	& 23	& 	24	& Y\\ 
	\hline
	y	& 24	& 	1	& B\\ 
	\hline
	z	& 25	& 	4	& E\\ 
	\hline
	\end{tabular} 	
\end{center}

En appliquant cette alphabet au texte chiffrée on retrouve le texte suivant :

\emph{C'est un trou de verdure ou chante une rivière accrochant follement aux herbes des haillons d'argent ou le soleil de la montagne fière luit c est un petit val qui mousse de rayons.}
         

	\end{enumerate}		

	%%%%%%%%%%%%%%%%%%%%%%%%%%%%%%%%%%%%%%%%%%%%%%%%%%%%%%%%%%%%%%%%%%%%%%%%%%%%%
%%%%%%%%%%%%%%%%%%%%%%%%%%%%%%%%%%%%%%%%%%%%%%%%%%%%%%%%%%%%%%%%%%%%%%%%%%%%%
%%%%%%%%%%%%%%%%%%%%%%%%%%%%%%%%%%%%%%%%%%%%%%%%%%%%%%%%%%%%%%%%%%%%%%%%%%%%%
	\exercice 
	
	On considère dans $F_2[X]$ le polynôme $P(X)=X^6+X^5+1$ et $A$ l'anneau $F_2[X]/(P(X))$.
	\begin{enumerate}
	\item
	Soit $\alpha$ la classe de $X$ dans $A$.Montrer qu $\alpha$ est une unité.
	
	$\alpha$ est une racine de $P(X)$, tel que :
	\begin{align}
		\alpha^6+\alpha^5+1=0 \notag \\
		\alpha^6+\alpha^5=1 \notag \\
		\alpha(\alpha^5+\alpha^4)=1 \notag 
	\end{align}
	
	On obtient une équation de la forme $\alpha x=1$ avec $x=\alpha^{-1}=\alpha^5+\alpha^4$. Donc $\alpha$ est inversible, on peut donc en déduire que $\alpha$ est une unité.
	
On cherche à calculer l'ordre d'$\alpha$.

\begin{center}

\begin{tabular}{|c|c|}
	\hline 	
	$\alpha^1$ & $\alpha^1$\\
	\hline 	
	$\alpha^2$ & $\alpha^2$\\
	\hline 	
	$\alpha^3$ & $\alpha^3$\\
	\hline 	
	$\alpha^4$ & $\alpha^4$\\
	\hline 	
	$\alpha^5$ & $\alpha^5$\\
	\hline 	
	$\alpha^6$ & $\alpha^5+1 $\\
	\hline 	
	$\alpha^7$ & $ \alpha^5+\alpha+1$\\
	\hline 	
	$\alpha^8$ &$\alpha^5+\alpha^2+\alpha+1 $ \\
	\hline 	
	$\alpha^9$ & $ \alpha^5+\alpha^3+\alpha^2+\alpha+1$\\
	\hline 	
	$\alpha^{10}$ & $ \alpha^5+\alpha^4+\alpha^3+\alpha^2+\alpha+1$\\
	\hline 	
	$\alpha^{11}$ & $\alpha^4+\alpha^3+\alpha^2+\alpha+1$\\
	\hline 	
	$\alpha^{12}$ & $ \alpha^5+\alpha^4+\alpha^3+\alpha^2+\alpha$\\
	\hline 	
	$\alpha^{13}$ & $ \alpha^4+\alpha^3+\alpha^2+1$\\
	\hline 	
	$\alpha^{14}$ & $ \alpha^5+\alpha^4+\alpha^3+\alpha$\\
	\hline 	
	$\alpha^{15}$ & $\alpha^4+\alpha^2+1 $\\
	\hline 	
	$\alpha^{16}$ & $ \alpha^5+\alpha^3+\alpha$\\
	\hline 	
	$\alpha^{17}$ & $ \alpha^5+\alpha^4+\alpha^2+1$\\
	\hline 	
	$\alpha^{18}$ & $\alpha^3+\alpha+1 $\\
	\hline 	
	$\alpha^{19}$ & $ \alpha^4+\alpha^2+\alpha$\\
	\hline 	
	$\alpha^{20}$ & $ \alpha^5+\alpha^3+\alpha^2$\\
	\hline 	
	$\alpha^{21}$ & $\alpha^5++\alpha^4+\alpha^3+1 $\\
	\hline
\end{tabular}
\end{center}

L'ordre $\alpha$ est un diviseur de l'ordre du groupe multiplicatif. L'anneau $A$ à $2^6$, c'est à dire 64 éléments. Le groupe multiplicatif à donc un ordre de 63. Le seul diviseur de 63 après 21 est 63. L'ordre d'$\alpha$ est donc 63.

\begin{center}
o$(\alpha)=63$
\end{center}

	\item
	L'ordre d'$\alpha$ est de 63. Il est donc générateur de tout les éléments du groupe multiplicatif. De plus, $\alpha$ est une unité, il est donc inversible. Les éléments du groupe générés par $\alpha$ sont donc inversible. Ces éléments appartiennent à $A$.Or, un anneau dont tout les éléments non nul sont inversibles est un corps. Donc A est un corps.
	\item
	L'ordre d'$\alpha$ est égal à l'ordre du groupe. De plus, $\alpha$ est une racine de $P(X)$, donc $\alpha$ est une racine primitive. Les racines d'un polynôme ont toutes le même ordre. Elles sont donc toutes primitives. Un polynôme primitif est un polynôme dont toutes les racines sont primitives. Ainsi, $P(X)$ est primitif.
	\item
Nombre de polynômes irréductibles de degré 6 dans $F_2[X]$.

$X^{2^6}+X=$produit des polynômes irréductible de degré divisant 6.

$X^{2^6}+X=X(X+1)(X^2+X+1)(X^3+X+1)(X^3+X+1)\times $produits de polynômes irréductible degré 6.

La somme des degré de tout les polynôme doit être égale à 64. Il y' a donc 9 polynômes irréductible de degré 6 dans $F_2[X]$.
	\item
Nombre de polynômes irréductibles primitifs de degré 6 dans $F_2[X]$ = $\frac{\varphi(2^n-1)}{n}$, avec $n=6$.


 $\frac{\varphi(2^6-1)}{6}=\frac{36}{6}=6$.
 
 Il y' a donc 6 polynômes irréductibles primitifs de degré 6 dans $F_2[X]$.

	\end{enumerate}	
%%%%%%%%%%%%%%%%%%%%%%%%%%%%%%%%%%%%%%%%%%%%%%%%%%%%%%%%%%%%%%%%%%%%%%%%%%%%%
%%%%%%%%%%%%%%%%%%%%%%%%%%%%%%%%%%%%%%%%%%%%%%%%%%%%%%%%%%%%%%%%%%%%%%%%%%%%%
%%%%%%%%%%%%%%%%%%%%%%%%%%%%%%%%%%%%%%%%%%%%%%%%%%%%%%%%%%%%%%%%%%%%%%%%%%%%%
	\exercice
	\begin{enumerate}
	\item
	Décomposer 341 en produit de facteurs premier :
	
	$341=11 \times 31$
	\item
	Montrer d'après le petit théorème de Fermat que $341$ n'est pas  premier.
	
	On cherche donc à calculer $a^{340}$, pour $a \in Z$.	
	On choisit $a=2$.
	
	\begin{align}
		2^{340} &\equiv (2^{10})^3 \times 2^4 \pmod{341}\notag \\
			  &\equiv 1024^3 \times 16 \pmod{341}\notag \\
			  &\equiv 1^3 \times 16 \pmod{341}\notag \\
			  &\equiv 16 \pmod{341}\notag
	\end{align}
	
	Ainsi, $2^{340}$ n'est pas congru à $1 \pmod{341}$, $341$ n'est pas premier.
	\end{enumerate}
		
	 \end{document}
		