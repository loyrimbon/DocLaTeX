\documentclass[a4paper,10pt]{article}

\usepackage[utf8]{inputenc}
\usepackage[T1]{fontenc}
\usepackage[francais]{babel}
\usepackage{amsthm}
\usepackage{amsmath}
\usepackage{graphics}
\usepackage{amssymb}
\usepackage{color}
\usepackage[table,xcdraw]{xcolor}
\usepackage{tikz}

\usepackage{xlop}
\usepackage{ifthen}
\usepackage[top=2cm,bottom=2cm,left=2cm,right=2cm]{geometry}

%\usepackage{mathsfs}

%\title{Devoir 1 de Mathématiques}
%\author{\textsc{RIMBON} Loy}
%\date{\today}

%!!!!!!!!!!!!!!!!!!!!!!!!!!!!               !!!!!!!!!!!!!!!!!!!!!!!!!!!!
%!!!!!!!!!!!!!!!!!!!!!!!!!!!!   Exercices   !!!!!!!!!!!!!!!!!!!!!!!!!!!!
%!!!!!!!!!!!!!!!!!!!!!!!!!!!!               !!!!!!!!!!!!!!!!!!!!!!!!!!!!

\newcounter{exercice}
\setcounter{exercice}{0}
\newcommand{\exercice}{%
	\refstepcounter{exercice}%
	\bigskip
	\bigskip
	\noindent\textbf{Exercice \theexercice.}~%
}
%%%%%%%%%%%%%%%%%%%%%%%%%%%%%%%%%%%%%%%%%%%%%%%%%%%%%%%
% styles TikZ pour les blocks
\tikzstyle{block}=[draw,rectangle,fill=white,inner sep=0mm,minimum size=12mm]
\tikzstyle{emptyblock}=[rectangle,fill=white,inner sep=0mm,minimum size=12mm]
\tikzstyle{sumg}=[draw,circle,node distance=1cm,inner sep=0mm,minimum size=10mm]
\tikzstyle{sum}=[draw,circle,node distance=1cm,inner sep=0mm,minimum size=6mm]
\tikzstyle{input}=[coordinate]
%%%%%%%%%%%%%%%%%%%%%%%%%%%%%%%%%%%%%%%%%%%%%%%%%%%%%%%
%!!!!!!!!!!!!!!!!!!!                                !!!!!!!!!!!!!!!!!!!
%!!!!!!!!!!!!!!!!!!!   Vos commandes personnelles   !!!!!!!!!!!!!!!!!!!
%!!!!!!!!!!!!!!!!!!!                                !!!!!!!!!!!!!!!!!!!

\newcommand{\stagiaire}{Loy RIMBON}
\newcommand{\datedevoir}{6 octobre 2016}
\newcommand{\titre}{Devoir \no 6}

\newcommand{\R}{\mathbb{R}}
\newcommand{\C}{\mathbb{C}}
\newcommand{\N}{\mathbb{N}}
\newcommand{\Z}{\mathbb{Z}}
\newcommand{\Q}{\mathbb{Q}}
\newcommand{\LCI}{$E\times E\longrightarrow E$ ~}

\renewcommand{\leq}{\leqslant}
\renewcommand{\geq}{\geqslant}

\newcounter{arret}%
\setcounter{arret}{0}%
\newcommand{\PGCD}[2]%
{%
\opcopy{#1}{a}%
\opcopy{#2}{b}%
\opcopy{#1}{A}%
\opcopy{#2}{B}%
\opgcd{A}{B}{PGCD}%
\noindent%
%Calculons par l'algorithme d'\textsc{Euclide} le PGCD des nombres $ \opprint{A} $ et $ \opprint{B} $.\\%
%\begin{align}


\whiledo{\equal{\thearret}{0}}%
{\opidiv*{a}{b}{q}{r}%
$ \opprint{a} = \opprint{b} \times \opprint{q} + \opprint{r} $\\
%\opprint{a} $= \opprint{b} \times \opprint{q} + \opprint{r} \notag   \\
\opcmp{r}{0}%
\ifopeq%
        \refstepcounter{arret}%
\fi%
%\end{align}
\opcopy{b}{a}%
\opcopy{r}{b}}%
%Le PGCD des nombres $ \opprint{A} $ et $ \opprint{B} $ est le dernier reste non nul du procédé,
%c'est-à-dire $ \opprint{PGCD} $.%
}%

\begin{document}
%\maketitle

\noindent\stagiaire\hfill\datedevoir

\bigskip
\bigskip
\begin{center}
{\large\bfseries\titre}
\end{center}
\bigskip
\bigskip

%%%%%%%%%%%%%%%%%%%%%%%%%%%%%%%%%%%%%%%%%%%%%%%%%%%%%%%%%%%%%%%%%%%%%%%%%%%%%
%%%%%%%%%%%%%%%%%%%%%%%%%%%%%%%%%%%%%%%%%%%%%%%%%%%%%%%%%%%%%%%%%%%%%%%%%%%%%
%%%%%%%%%%%%%%%%%%%%%%%%%%%%%%%%%%%%%%%%%%%%%%%%%%%%%%%%%%%%%%%%%%%%%%%%%%%%%

%%%%%%%%%%%%%%%%%%%%%%%%%%%%%%%%%%%%%%%%%%%%%%%%%%%%%%%%%%%%%%%%%%%%%%%%%%%%%
%%%%%%%%%%%%%%%%%%%%%%%%%%%%%%%%%%%%%%%%%%%%%%%%%%%%%%%%%%%%%%%%%%%%%%%%%%%%%
%%%%%%%%%%%%%%%%%%%%%%%%%%%%%%%%%%%%%%%%%%%%%%%%%%%%%%%%%%%%%%%%%%%%%%%%%%%%%
\exercice 

\begin{enumerate}

%QUESTION 1
\item
On cherche à démontrer la proposition suivante : si $a \wedge b=1$, alors $a \wedge a+b=1$.

Soit $d$ le $pgcd(a,a+b)$ tel que $d \neq 1$. Il existe $u$ et $v \in \Z^2$ tel que $au+(a+b)v=d$.
\begin{align}
	au+(a+b)v&=d \notag \\
	au+av+bv &= d \notag \\
	a(\underbrace{u+v}_{\in \Z}) + bv &=d \notag 
\end{align}


La contraposée de la proposition et vérifiée. Ainsi, si $a \wedge b=1$, alors $a \wedge a+b=1$.  
%QUESTION 2
\item
	$pgcd(a,b)=1 $ et $pgcd(a,c)=1$. Soit $(u,v,u',v') \in \Z^4 $ tel que $au+bv=1$ et $au'+cv'=1$
	\begin{align}
		(au+bv)\times(au'+cv')&=1 \notag \\
		auau'+aucv'+bvau'+bvcv'&=1 \notag \\
		a(\underbrace{auu'+ucv'+bvu'}_{\in \Z})+bc\underbrace{cv'}_{\in \Z} &=1 \notag
	\end{align}

	Il existe donc un couple $(u,v) \in \Z $ tel que $au+bcv=1$.

	\end{enumerate}

%%%%%%%%%%%%%%%%%%%%%%%%%%%%%%%%%%%%%%%%%%%%%%%%%%%%%%%%%%%%%%%%%%%%%%%%%%%%%
%%%%%%%%%%%%%%%%%%%%%%%%%%%%%%%%%%%%%%%%%%%%%%%%%%%%%%%%%%%%%%%%%%%%%%%%%%%%%
%%%%%%%%%%%%%%%%%%%%%%%%%%%%%%%%%%%%%%%%%%%%%%%%%%%%%%%%%%%%%%%%%%%%%%%%%%%%%
\exercice 

\begin{enumerate}

%QUESTION 1
\item
	Déterminer le PGCD $d$ de 2412 et de 670.

	\begin{align}
		2412 &= 670 \times 3 + 402 \notag \\
		670 &=  402 \times 1 + 268 \notag \\
		402 &= 268 \times 1 + 134 \notag \\
		268 &= 134 \times 2 + 0 \notag
	\end{align}
	Ainsi, $2412 \wedge 670 = 134$

%QUESTION 2
\item
	Déterminer l'ensemble des couples $(u, v)$ de $\Z^2$ tels que $2412u + 670v=d$
	\begin{center}
		$2412u + 670v=134 \Longleftrightarrow18u+5v=1$
	\end{center}
	
	On vérifie que $18 \wedge 5=1$

	\begin{align}
		18 &= 5 \times 3 + 3 \notag \\
		5 &= 3 \times 1 + 2 \notag \\
		3 &= 2 \times 1 + 1 \notag \\
		2 &= 1 \times 2 + 0 \notag
	\end{align}
On cherche maintenant à déterminer $u$ et $v$ en utilisant les équations précédentes.
	\begin{align}
		1 &= 3-2 \times 1 \notag \\
		&= 3-(5-3) \notag \\
		&= 18-5 \times 3 -(5-(18-5 \times 3)) \notag \\
		&=18-5\times 3-(5-18+5\times 3) \notag \\
		&=18 - 5 \times 3 -(5 \times 4 - 18) \notag \\
		&= 18-5 \times -5 \times 4 + 18 \notag \\
		&= 18 \times 2 - 5 \times 7 \notag
	\end{align}

	On a donc déterminer $u=2$ et $v=-7$. On peut à présent essayer de déterminer l'ensemble des couples $(u,v)$ solution de l'équation de départ.

	\begin{align}
		18u + 5V &=1 \\	
		18 \times 2 - 5 \times 7 = 1 
	\end{align}

	En soustrayant les équations 1 et 2, on obtient l'équation suivante :
	\begin{align}
		18(u-2)+5(v+7)&=0 \notag \\
		5(v+7) &= 18(2-u)  
	\end{align}

Ainsi, $5$ divise $18(2-u)$ et $5 \wedge 18=1 \implies 5$ divise $(2-u)$.

Ainsi, $(2-u)=5k$ et $ k \in \Z$, donc $u = 2-5k$.

On injecte dans l'équation (3) la nouvelle expression de $u$.
	\begin{align}
		5(v+7) &= 18(2 - 2+5k) \notag \\
		5(v + 7) &= 18 \times 5k \notag \\
		v+7 &= 18k \notag
	\end{align}

\[ solutions
\left\{
\begin{aligned}	
	v=18k-7\notag \\
	u=2-5k \notag \\	
\end{aligned}
\right.\]
%\opsub[carrysub]{18u+5v}{18 2 - 5 7}
\end{enumerate}

Vérification :
 $k=7$ alors $u=-33 $ et $v=119$.
\begin{align}
	18 \times -33 + 5 \times 119 &= 1 \notag \\
	2412 \times -33 + 670 \times 119 &= 134 \notag
\end{align}

L'ensemble des couples $(u,v)$ tels que $2412u + 670v = d$ est $(2-5k,18k-7)$, $ k \in \Z$.

%%%%%%%%%%%%%%%%%%%%%%%%%%%%%%%%%%%%%%%%%%%%%%%%%%%%%%%%%%%%%%%%%%%%%%%%%%%%%
%%%%%%%%%%%%%%%%%%%%%%%%%%%%%%%%%%%%%%%%%%%%%%%%%%%%%%%%%%%%%%%%%%%%%%%%%%%%%
%%%%%%%%%%%%%%%%%%%%%%%%%%%%%%%%%%%%%%%%%%%%%%%%%%%%%%%%%%%%%%%%%%%%%%%%%%%%%

%\begin{center}
	%\PGCD{1895664}{47856}
%	\opgcd{1895664}{47856}{d}
%	$\opprint{d}$.
%\end{center}

 \end{document}
