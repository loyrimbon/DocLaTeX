 	\documentclass[a4paper,10pt]{article}
	
	\usepackage[utf8]{inputenc}
	\usepackage[T1]{fontenc}
	\usepackage[francais]{babel}
	\usepackage{amsthm}
	\usepackage{amsmath}
	\usepackage{graphics}
	\usepackage{amssymb}
	\usepackage{color}
	\usepackage[table,xcdraw]{xcolor}
	\usepackage{tikz}
	
	\usepackage{xlop}
	\usepackage{ifthen}
	\usepackage[top=2cm,bottom=2cm,left=2cm,right=2cm]{geometry}

	%\usepackage{mathsfs}
	
	%\title{Devoir 1 de Mathématiques}
	%\author{\textsc{RIMBON} Loy}
	%\date{\today}
	
	%!!!!!!!!!!!!!!!!!!!!!!!!!!!!               !!!!!!!!!!!!!!!!!!!!!!!!!!!!
	%!!!!!!!!!!!!!!!!!!!!!!!!!!!!   Exercices   !!!!!!!!!!!!!!!!!!!!!!!!!!!!
	%!!!!!!!!!!!!!!!!!!!!!!!!!!!!               !!!!!!!!!!!!!!!!!!!!!!!!!!!!
	
	\newcounter{exercice}
	\setcounter{exercice}{0}
	\newcommand{\exercice}{%
		\refstepcounter{exercice}%
		\bigskip
		\bigskip
		\noindent\textbf{Exercice \theexercice.}~%
	}
	%%%%%%%%%%%%%%%%%%%%%%%%%%%%%%%%%%%%%%%%%%%%%%%%%%%%%%%
	% styles TikZ pour les blocks
	\tikzstyle{block}=[draw,rectangle,fill=white,inner sep=0mm,minimum size=12mm]
	\tikzstyle{emptyblock}=[rectangle,fill=white,inner sep=0mm,minimum size=12mm]
	\tikzstyle{sumg}=[draw,circle,node distance=1cm,inner sep=0mm,minimum size=10mm]
	\tikzstyle{sum}=[draw,circle,node distance=1cm,inner sep=0mm,minimum size=6mm]
	\tikzstyle{input}=[coordinate]
	%%%%%%%%%%%%%%%%%%%%%%%%%%%%%%%%%%%%%%%%%%%%%%%%%%%%%%%
	%!!!!!!!!!!!!!!!!!!!                                !!!!!!!!!!!!!!!!!!!
	%!!!!!!!!!!!!!!!!!!!   Vos commandes personnelles   !!!!!!!!!!!!!!!!!!!
	%!!!!!!!!!!!!!!!!!!!                                !!!!!!!!!!!!!!!!!!!
	
	\newcommand{\stagiaire}{Loy RIMBON}
	\newcommand{\datedevoir}{8 novembre 2016}
	\newcommand{\titre}{Devoir \no 9}
	
	\newcommand{\R}{\mathbb{R}}
	\newcommand{\C}{\mathbb{C}}
	\newcommand{\N}{\mathbb{N}}
	\newcommand{\Z}{\mathbb{Z}}
	\newcommand{\Q}{\mathbb{Q}}
	\newcommand{\LCI}{$E\times E\longrightarrow E$ ~}
	
	\renewcommand{\leq}{\leqslant}
	\renewcommand{\geq}{\geqslant}
	
	\newcounter{arret}%
\setcounter{arret}{0}%
\newcommand{\PGCD}[2]%
{%
\opcopy{#1}{a}%
\opcopy{#2}{b}%
\opcopy{#1}{A}%
\opcopy{#2}{B}%
\opgcd{A}{B}{PGCD}%
\noindent%
%Calculons par l'algorithme d'\textsc{Euclide} le PGCD des nombres $ \opprint{A} $ et $ \opprint{B} $.\\%
\whiledo{\equal{\thearret}{0}}%
{\opidiv*{a}{b}{q}{r}%
$ \opprint{a} = \opprint{b} \times \opprint{q} + \opprint{r} $\\%
\opcmp{r}{0}%
\ifopeq%
        \refstepcounter{arret}%
\fi%
\opcopy{b}{a}%
\opcopy{r}{b}}%
%Le PGCD des nombres $ \opprint{A} $ et $ \opprint{B} $ est le dernier reste non nul du procédé,
%c'est-à-dire $ \opprint{PGCD} $.%
}%
	
	\begin{document}
	%\maketitle
	
	\noindent\stagiaire\hfill\datedevoir
	
	\bigskip
	\bigskip
	\begin{center}
	{\large\bfseries\titre}
	\end{center}
	\bigskip
	\bigskip
	
	%%%%%%%%%%%%%%%%%%%%%%%%%%%%%%%%%%%%%%%%%%%%%%%%%%%%%%%%%%%%%%%%%%%%%%%%%%%%%
%%%%%%%%%%%%%%%%%%%%%%%%%%%%%%%%%%%%%%%%%%%%%%%%%%%%%%%%%%%%%%%%%%%%%%%%%%%%%
%%%%%%%%%%%%%%%%%%%%%%%%%%%%%%%%%%%%%%%%%%%%%%%%%%%%%%%%%%%%%%%%%%%%%%%%%%%%%
	\exercice 
	
	\begin{enumerate}
	 \item Une lettre est un p-uplet formé à partir des deux symboles . et -. quel est le nombre de "lettres" de longueur inférieur ou égale à p.
	 
	 \begin{center}
	 	 $\displaystyle\sum_{i=1}^p 2^i$
	 \end{center}
	
	 \item Nombre de pièces dans un jeu de domino.
	 \begin{center}
		$\displaystyle\sum_{i=7}^1 i = 28$
	 \end{center}
	 \item $2n$ personnes prennent place autour d'un table ronde.
	 \begin{center}
	 	$2n(2n-1)~!$
	  \end{center}
	 De combien de façon peuvent elle s'asseoir :
	 
	 Il y' a $n $ hommes et $n $ femmes. De combien de façon peuvent elles s'asseoir en respectant l'alternance.
	  \begin{center}
	 	 $(n~!)^2$
	  \end{center}
	\end{enumerate}
	
		%%%%%%%%%%%%%%%%%%%%%%%%%%%%%%%%%%%%%%%%%%%%%%%%%%%%%%%%%%%%%%%%%%%%%%%%%%%%%
%%%%%%%%%%%%%%%%%%%%%%%%%%%%%%%%%%%%%%%%%%%%%%%%%%%%%%%%%%%%%%%%%%%%%%%%%%%%%
%%%%%%%%%%%%%%%%%%%%%%%%%%%%%%%%%%%%%%%%%%%%%%%%%%%%%%%%%%%%%%%%%%%%%%%%%%%%%
	\exercice Soit $E$ un ensemble de $n$ personnes (n>2). Chacune d'entre elles envoie une lettre et une seule à une autre personne.
	
	\begin{enumerate}
	\item
	Chaque personne à $(n-1)$ destination possible, il y'a donc $(n-1)^n$ manières différentes d' adressées les $n$ lettres.
	\item Une personne A est désignée d'avance, on note $N_j$ le nombre de manières de distribuer les $n$ lettres de telle sorte que A reçoive exactement $j $ lettres $(0\leq j \leq n)$. 
	
	Nombre de manière d'adresser $j$ lettre à $A$ : 
		$\binom{ n-1}{j}$
		
	Nombre de manière pour $A$ d'envoyer sa lettre : $(n-1)$
	
	Nombre de manière pour les autres (ni $A$, ni les $j$ personnes) d'envoyer une lettre : $(n-2)^{(n-j-1)}$
	
	Ainsi, on obtient :
	\begin{center}
		$N_j = \binom{ n-1}{j}(n-2)^{(n-j-1)}(n-1)$
	 \end{center}
	\end{enumerate}
	
	
%%%%%%%%%%%%%%%%%%%%%%%%%%%%%%%%%%%%%%%%%%%%%%%%%%%%%%%%%%%%%%%%%%%%%%%%%%%%%
%%%%%%%%%%%%%%%%%%%%%%%%%%%%%%%%%%%%%%%%%%%%%%%%%%%%%%%%%%%%%%%%%%%%%%%%%%%%%
%%%%%%%%%%%%%%%%%%%%%%%%%%%%%%%%%%%%%%%%%%%%%%%%%%%%%%%%%%%%%%%%%%%%%%%%%%%%%
	\exercice
	\begin{enumerate}
	\item Un joueur tire 12 cartes d'un jeu de 52 cartes. Déterminer le nombre de tirages possible tels que :
	
		\begin{enumerate}
		\item Il y' ait quatre as : 
		On positionne les 4 as :
		$\binom{12}{4}$
		
		Nombre de façon de tirer quatre as :
		$\binom{4}{4}$
		
		On les place :$4!$
		
		
		Nombre de façon de tirer 8 autres cartes :
		$\binom{48}{8}$
		
		On places les 8 cartes : $8!$ 	
			
		\textbf{Résultat :}
		\begin{center}
		$\binom{ 12}{4}$
		$\binom{ 4}{4}$
		$4!$
		$\binom{ 48}{8}$
		$8!$ 
		\end{center}
		
		\medskip
		
		\item quatre as tirés successivement : 
		
		On positionne le 1er as : 9 possibilités
		
		Nombre de façon de tirer quatre as :
		$\binom{ 4}{4}$
		
		Nombre de façon de les disposer :$4!$
		
		Nombre de façon de tirer 8 autres cartes :
		$\binom{48}{8}$
		
		On place les cartes : $8!$
		
		\newpage
		
		\textbf{Résultat :}
		\begin{center}
		$9\binom{4}{4}$
		$4!$
		$\binom{48}{8}$
		$8!$
		\end{center}
		
		\item on obtienne 7 piques et 2 dames :  
		
		\textbf{1er cas :7 piques et 2 dames}
		
		On positionne les 7 piques :$\binom{12}{7}$
		
		On les choisit : $\binom{12}{7} $
		
		On les places :$7!$
		
		On positionnes les 2 dames : 
		$\binom{5}{2}$
		
		On choisit les 2 dames:
		$\binom{3}{2}$
		
		On les places :$2!$
		
		On choisit les 3 cartes restante :
		$\binom{36}{3}$
		
		On les places : $3!$
		
		
		\textbf{2ème cas : 6 piques, 2 dames dont 1 dame de pique}
		
		On positionne les 6 piques:$\binom{12}{6}$
		
		On les choisit :$\binom{12}{6}$
		
		On les places :$6!$
		
		On positionne la dame de pique : 6 possibilités
		
		On la choisit : $\binom{1}{1}$
		
		On positionne la dame : 5 possibilités
		
		On choisit une dame :$\binom{3}{1}$
				
		On choisit les 4 cartes restantes :$\binom{36}{4}$		
		
		On les places : $4!$	
		
		
		\textbf{Ce qui nous donne :}
		
		$\binom{12}{7}$
		$\binom{12}{7}$
		$7!$
		$\binom{5}{2}$
		$\binom{3}{2}$
		$2!$
		$\binom{36}{3}$
		$3!+$
		$\binom{12}{6}$
		$\binom{12}{6}$
		$6!$
		$6$
		$\binom{1}{1}$
		$\binom{3}{1}$
		$\binom{36}{4}$		
		$4!$	
		
		
		\end{enumerate}
	
	\item Un joueur tire successivement et avec remise 8 cartes d'un jeu de 32 cartes. Déterminer le nombre de tirages possible tels que :
		\begin{enumerate}
		\item on obtienne 2 piques, 2 cœurs et 4 trèfles

		
		On choisit la position du pique :
		$\binom{8}{1}$
		
		On choisit le pique :
		$\binom{8}{1}$
		
		On choisit la position du 2ème pique :
		$\binom{7}{1}$
		
		On choisit le 2ème pique :
		$\binom{8}{1}$
		
		On choisit la position du coeur :
		$\binom{6}{1}$
		
		On choisit le coeur :
		$\binom{8}{1}$
		
		On choisit la position du 2ème coeur :
		$\binom{5}{1}$
		
		On choisit le 2ème coeur :
		$\binom{8}{1}$
		
		On choisit la position du 1er trèfle :
		$\binom{4}{1}$
		
		On choisit les 1er trèfles :		
		$\binom{8}{1}$
		
		On choisit la position du 2ème trèfle :
		$\binom{3}{1}$
		
		On choisit les 2ème trèfles :		
		$\binom{8}{1}$
		
		On choisit la position du 3ème trèfle :
		$\binom{2}{1}$
		
		On choisit le 3ème trèfles :		
		$\binom{8}{1}$
		
		On choisit la position du 4ème trèfle :
		$\binom{1}{1}$
		
		On choisit le 4ème trèfles :		
		$\binom{8}{1}$
		
		\textbf{Ce qui nous donne :}
		
		$\binom{8}{1}$
		$\binom{8}{1}$
		$\binom{7}{1}$
		$\binom{8}{1}$
		$\binom{6}{1}$
		$\binom{8}{1}$
		$\binom{5}{1}$
		$\binom{8}{1}$
		$\binom{4}{1}$
		$\binom{8}{1}$
		$\binom{3}{1}$
		$\binom{8}{1}$
		$\binom{2}{1}$
		$\binom{8}{1}$
		$\binom{1}{1}$
		$\binom{8}{1}=$
		$8!8^8$
		
		\item on obtienne 2 piques et un roi exactement :
		
		Deux cas de figure : on tire 2 piques et un roi ou on tire un pique et un roi de pique:
		\textbf{1er cas :2 piques et un roi}
		
		On choisit la position du 1er pique :$\binom{8}{1}$
		
		Nombre de manière de choisir le 1er pique :$\binom{7}{1}$
		
		On choisit la position du 2ème pique :$\binom{7}{1}$
		
		Nombre de manière de choisit le 1er pique :$\binom{7}{1}$
		
		On choisit la position du 3 :$\binom{6}{1}$
		
		Nombre de manière de choisir le roi :$\binom{3}{1}$
		
		Il reste ensuite 5 cartes à choisir parmi les cœurs, trèfles et carreaux restant 				sans prendre de roi :
		
		$\binom{21}{5}\times5\times 4\times 3 \times 2\times1$
		
		\textbf{2ème cas : un pique et un roi de pique}
		
		On choisit la position du 1er pique :$\binom{8}{1}$
		
		Nombre de manière de choisit le 1er pique :$\binom{7}{1}$
		
		On choisit la position du roi :$\binom{7}{1}$
		
		Nombre de manière de choisir le roi de pique:$\binom{1}{1}$
		
		Il reste ensuite 5 cartes à choisir parmi les cœurs, trèfles et carreaux restant 				sans prendre de roi :
		
		$\binom{21}{6}\times 6\times 5\times 4\times 3\times 2\times 1$
		
		\textbf{Ce qui donne pour résultat :}
		
		$\binom{8}{1}$
		$\binom{7}{1}$
		$\binom{7}{1}$
		$\binom{1}{1}$
		$\binom{21}{6}$		
		$\binom{8}{1}$
		$\binom{7}{1}$
		$\binom{7}{1}$
		$\binom{7}{1}$
		$\binom{6}{1}$
		$\binom{3}{1}$
		$\binom{21}{5}\times5\times 4\times 3 \times 2\times1+$
		$\binom{8}{1}$
		$\binom{7}{1}$
		$\binom{7}{1}$
		$\binom{1}{1}$
		$\binom{21}{6}\times 6\times 5\times 4\times 3\times 2\times 1$
	
		
		\end{enumerate}
	\end{enumerate}
	
%%%%%%%%%%%%%%%%%%%%%%%%%%%%%%%%%%%%%%%%%%%%%%%%%%%%%%%%%%%%%%%%%%%%%%%%%%%%%
%%%%%%%%%%%%%%%%%%%%%%%%%%%%%%%%%%%%%%%%%%%%%%%%%%%%%%%%%%%%%%%%%%%%%%%%%%%%%
%%%%%%%%%%%%%%%%%%%%%%%%%%%%%%%%%%%%%%%%%%%%%%%%%%%%%%%%%%%%%%%%%%%%%%%%%%%%%
	\exercice
	Une urne contient $n$ boules numérotées de 1 à $n$. $n \in \N^*-\{1,2\}$. On tire successivement avec remise $q$ boules de cette urne.On note $x_i$ le numéro de la $i^e$ boule tirée. Déterminer le nombre de tirage tel que :
		\begin{enumerate}
		\item $x_i<x_q$ 
		\begin{center}
		$\binom{n}{1}$
		$\binom{n-x_1}{ 1}$
		 \end{center}
		\item la somme des numéros tirés est égale à $q+2$
		
		Un tirage comporte $q$ termes. Pour que la somme des $q$ termes soient égale à $q+2$ il faut que tout les termes soient égale à $1$ et que l'un des termes soient égale à 3 ou deux termes soient égales à 2.
		
		Le nombre de tirage est donc obtenue en additionnant les différentes façon de tirer le 3 ou les 2 :
		\begin{center}
			$\binom{q}{2}$
		 \end{center}
		\item deux numéros exactement au cours du tirage :
		nombre de couple de deux numéros différents  multipliés par le nombre de façon d'effectuer $q$ tirages avec seulement 2 nombres :
		\begin{center}
			$\frac{n(n-1)}{2}2^q-2 $
		 \end{center}
		
		\end{enumerate}
	 \end{document}
		