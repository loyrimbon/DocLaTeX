\documentclass[a4paper,10pt]{article}

\usepackage[utf8]{inputenc}
\usepackage[T1]{fontenc}
\usepackage[francais]{babel}
\usepackage{amsthm}
\usepackage{amsmath}
\usepackage{graphics}
\usepackage{amssymb}
\usepackage{color}
\usepackage[table,xcdraw]{xcolor}
\usepackage{tikz}

\usepackage{xlop}
\usepackage{ifthen}
\usepackage[top=2cm,bottom=2cm,left=2cm,right=2cm]{geometry}

%\usepackage{mathsfs}

%\title{Devoir 1 de Mathématiques}
%\author{\textsc{RIMBON} Loy}
%\date{\today}

%!!!!!!!!!!!!!!!!!!!!!!!!!!!!               !!!!!!!!!!!!!!!!!!!!!!!!!!!!
%!!!!!!!!!!!!!!!!!!!!!!!!!!!!   Exercices   !!!!!!!!!!!!!!!!!!!!!!!!!!!!
%!!!!!!!!!!!!!!!!!!!!!!!!!!!!               !!!!!!!!!!!!!!!!!!!!!!!!!!!!

\newcounter{exercice}
\setcounter{exercice}{0}
\newcommand{\exercice}{%
	\refstepcounter{exercice}%
	\bigskip
	\bigskip
	\noindent\textbf{Exercice \theexercice.}~%
}
%%%%%%%%%%%%%%%%%%%%%%%%%%%%%%%%%%%%%%%%%%%%%%%%%%%%%%%
% styles TikZ pour les blocks
\tikzstyle{block}=[draw,rectangle,fill=white,inner sep=0mm,minimum size=12mm]
\tikzstyle{emptyblock}=[rectangle,fill=white,inner sep=0mm,minimum size=12mm]
\tikzstyle{sumg}=[draw,circle,node distance=1cm,inner sep=0mm,minimum size=10mm]
\tikzstyle{sum}=[draw,circle,node distance=1cm,inner sep=0mm,minimum size=6mm]
\tikzstyle{input}=[coordinate]
%%%%%%%%%%%%%%%%%%%%%%%%%%%%%%%%%%%%%%%%%%%%%%%%%%%%%%%
%!!!!!!!!!!!!!!!!!!!                                !!!!!!!!!!!!!!!!!!!
%!!!!!!!!!!!!!!!!!!!   Vos commandes personnelles   !!!!!!!!!!!!!!!!!!!
%!!!!!!!!!!!!!!!!!!!                                !!!!!!!!!!!!!!!!!!!

\newcommand{\stagiaire}{Loy RIMBON}
\newcommand{\datedevoir}{10 octobre 2016}
\newcommand{\titre}{Devoir \no 7}

\newcommand{\R}{\mathbb{R}}
\newcommand{\C}{\mathbb{C}}
\newcommand{\N}{\mathbb{N}}
\newcommand{\Z}{\mathbb{Z}}
\newcommand{\Q}{\mathbb{Q}}
\newcommand{\LCI}{$E\times E\longrightarrow E$ ~}

\renewcommand{\leq}{\leqslant}
\renewcommand{\geq}{\geqslant}

\newcounter{arret}%
\setcounter{arret}{0}%
\newcommand{\PGCD}[2]%
{%
\opcopy{#1}{a}%
\opcopy{#2}{b}%
\opcopy{#1}{A}%
\opcopy{#2}{B}%
\opgcd{A}{B}{PGCD}%
\noindent%
%Calculons par l'algorithme d'\textsc{Euclide} le PGCD des nombres $ \opprint{A} $ et $ \opprint{B} $.\\%
%\begin{align}


\whiledo{\equal{\thearret}{0}}%
{\opidiv*{a}{b}{q}{r}%
$ \opprint{a} = \opprint{b} \times \opprint{q} + \opprint{r} $\\
%\opprint{a} $= \opprint{b} \times \opprint{q} + \opprint{r} \notag   \\
\opcmp{r}{0}%
\ifopeq%
        \refstepcounter{arret}%
\fi%
%\end{align}
\opcopy{b}{a}%
\opcopy{r}{b}}%
%Le PGCD des nombres $ \opprint{A} $ et $ \opprint{B} $ est le dernier reste non nul du procédé,
%c'est-à-dire $ \opprint{PGCD} $.%
}%

\begin{document}
%\maketitle

\noindent\stagiaire\hfill\datedevoir

\bigskip
\bigskip
\begin{center}
{\large\bfseries\titre}
\end{center}
\bigskip
\bigskip

%%%%%%%%%%%%%%%%%%%%%%%%%%%%%%%%%%%%%%%%%%%%%%%%%%%%%%%%%%%%%%%%%%%%%%%%%%%%%
%%%%%%%%%%%%%%%%%%%%%%%%%%%%%%%%%%%%%%%%%%%%%%%%%%%%%%%%%%%%%%%%%%%%%%%%%%%%%
%%%%%%%%%%%%%%%%%%%%%%%%%%%%%%%%%%%%%%%%%%%%%%%%%%%%%%%%%%%%%%%%%%%%%%%%%%%%%

%%%%%%%%%%%%%%%%%%%%%%%%%%%%%%%%%%%%%%%%%%%%%%%%%%%%%%%%%%%%%%%%%%%%%%%%%%%%%
%%%%%%%%%%%%%%%%%%%%%%%%%%%%%%%%%%%%%%%%%%%%%%%%%%%%%%%%%%%%%%%%%%%%%%%%%%%%%
%%%%%%%%%%%%%%%%%%%%%%%%%%%%%%%%%%%%%%%%%%%%%%%%%%%%%%%%%%%%%%%%%%%%%%%%%%%%%
\exercice 



%%%%%%%%%%%%%%%%%%%%%%%%%%%%%%%%%%%%%%%%%%%%%%%%%%%%%%%%%%%%%%%%%%%%%%%%%%%%%
%%%%%%%%%%%%%%%%%%%%%%%%%%%%%%%%%%%%%%%%%%%%%%%%%%%%%%%%%%%%%%%%%%%%%%%%%%%%%
%%%%%%%%%%%%%%%%%%%%%%%%%%%%%%%%%%%%%%%%%%%%%%%%%%%%%%%%%%%%%%%%%%%%%%%%%%%%%
\exercice 

%%%%%%%%%%%%%%%%%%%%%%%%%%%%%%%%%%%%%%%%%%%%%%%%%%%%%%%%%%%%%%%%%%%%%%%%%%%%%
%%%%%%%%%%%%%%%%%%%%%%%%%%%%%%%%%%%%%%%%%%%%%%%%%%%%%%%%%%%%%%%%%%%%%%%%%%%%%
%%%%%%%%%%%%%%%%%%%%%%%%%%%%%%%%%%%%%%%%%%%%%%%%%%%%%%%%%%%%%%%%%%%%%%%%%%%%%
\exercice
Résoudre dans $\Z$

\[
\left\{
\begin{aligned}	
	x \equiv 5(mod 12) (1)\\
	x \equiv 3(mod 35) (2)\\	
	x \equiv 8(mod 13) (3)
	\end{aligned}	
\right.\]

\PGCD{35}{13}

Le $pgcd(35,13)=1$, donc il existe une unique solution modulo $35 \times 13)$ pour (2) et (3).

$x=35k +3 \equiv 9k+3(mod 13)=8(mod)13$

\begin{align}
	9k &= 5(mod 13) \notag \\
	3 \times 9k &= 3 \times 5(mod 13) \notag \\
	k &=2(mod 13) \notag \\
	k &= 13k'+2 \notag
\end{align}

\begin{align}
	x &= 35(13k'+2)+3 \notag \\
	   &= 35 \times 13k' +73 \notag \\
	   &\equiv 73(mod 35 \times 13) \notag
\end{align}

\[
\left\{
\begin{aligned}	
	x \equiv 73(mod 455) (4)\\	
	x \equiv 8(mod 13) (5)
	\end{aligned}	
\right.\]

Le $pgcd(73,5)=1$. Ainsi il existe une unique solution modulo $455 \times 13 $ pour (4) et (5).

$x=455k +73 \equiv 11k + 1 (mod 12)=5 (mod 12)$

\begin{align}
	11k&=4(mod 12) \notag \\
	11 \times 11k &= 44 (mod 12) \notag \\	
	k &= 8 (mod 12) \notag \\
	k &= 12k' +8 \notag
\end{align}

\begin{align}
	x &= 455(12k'+8)+73 \notag \\
	   &= 455 \times 12k' + 3640 + 73 \notag \\
	   &= 455 \times 12k'+3713 \notag \\
	   &\equiv 3713 (mod 455 \times 12) \notag 
\end{align}
 \end{document}
	