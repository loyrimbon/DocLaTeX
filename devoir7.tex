	\documentclass[a4paper,10pt]{article}
	
	\usepackage[utf8]{inputenc}
	\usepackage[T1]{fontenc}
	\usepackage[francais]{babel}
	\usepackage{amsthm}
	\usepackage{amsmath}
	\usepackage{graphics}
	\usepackage{amssymb}
	\usepackage{color}
	\usepackage[table,xcdraw]{xcolor}
	\usepackage{tikz}
	
	\usepackage{xlop}
	\usepackage{ifthen}
	\usepackage[top=2cm,bottom=2cm,left=2cm,right=2cm]{geometry}
	
	%\usepackage{mathsfs}
	
	%\title{Devoir 1 de Mathématiques}
	%\author{\textsc{RIMBON} Loy}
	%\date{\today}
	
	%!!!!!!!!!!!!!!!!!!!!!!!!!!!!               !!!!!!!!!!!!!!!!!!!!!!!!!!!!
	%!!!!!!!!!!!!!!!!!!!!!!!!!!!!   Exercices   !!!!!!!!!!!!!!!!!!!!!!!!!!!!
	%!!!!!!!!!!!!!!!!!!!!!!!!!!!!               !!!!!!!!!!!!!!!!!!!!!!!!!!!!
	
	\newcounter{exercice}
	\setcounter{exercice}{0}
	\newcommand{\exercice}{%
		\refstepcounter{exercice}%
		\bigskip
		\bigskip
		\noindent\textbf{Exercice \theexercice.}~%
	}
	%%%%%%%%%%%%%%%%%%%%%%%%%%%%%%%%%%%%%%%%%%%%%%%%%%%%%%%
	% styles TikZ pour les blocks
	\tikzstyle{block}=[draw,rectangle,fill=white,inner sep=0mm,minimum size=12mm]
	\tikzstyle{emptyblock}=[rectangle,fill=white,inner sep=0mm,minimum size=12mm]
	\tikzstyle{sumg}=[draw,circle,node distance=1cm,inner sep=0mm,minimum size=10mm]
	\tikzstyle{sum}=[draw,circle,node distance=1cm,inner sep=0mm,minimum size=6mm]
	\tikzstyle{input}=[coordinate]
	%%%%%%%%%%%%%%%%%%%%%%%%%%%%%%%%%%%%%%%%%%%%%%%%%%%%%%%
	%!!!!!!!!!!!!!!!!!!!                                !!!!!!!!!!!!!!!!!!!
	%!!!!!!!!!!!!!!!!!!!   Vos commandes personnelles   !!!!!!!!!!!!!!!!!!!
	%!!!!!!!!!!!!!!!!!!!                                !!!!!!!!!!!!!!!!!!!
	
	\newcommand{\stagiaire}{Loy RIMBON}
	\newcommand{\datedevoir}{10 octobre 2016}
	\newcommand{\titre}{Devoir \no 7}
	
	\newcommand{\R}{\mathbb{R}}
	\newcommand{\C}{\mathbb{C}}
	\newcommand{\N}{\mathbb{N}}
	\newcommand{\Z}{\mathbb{Z}}
	\newcommand{\Q}{\mathbb{Q}}
	\newcommand{\LCI}{$E\times E\longrightarrow E$ ~}
	
	\renewcommand{\leq}{\leqslant}
	\renewcommand{\geq}{\geqslant}
	
	\newcounter{arret}%
\setcounter{arret}{0}%
\newcommand{\PGCD}[2]%
{%
\opcopy{#1}{a}%
\opcopy{#2}{b}%
\opcopy{#1}{A}%
\opcopy{#2}{B}%
\opgcd{A}{B}{PGCD}%
\noindent%
%Calculons par l'algorithme d'\textsc{Euclide} le PGCD des nombres $ \opprint{A} $ et $ \opprint{B} $.\\%
%\begin{align}


\whiledo{\equal{\thearret}{0}}%
{\opidiv*{a}{b}{q}{r}%
$ \opprint{a} = \opprint{b} \times \opprint{q} + \opprint{r} $\\
%\opprint{a} $= \opprint{b} \times \opprint{q} + \opprint{r} \notag   \\
\opcmp{r}{0}%
\ifopeq%
        \refstepcounter{arret}%
\fi%
%\end{align}
\opcopy{b}{a}%
\opcopy{r}{b}}%
%Le PGCD des nombres $ \opprint{A} $ et $ \opprint{B} $ est le dernier reste non nul du procédé,
%c'est-à-dire $ \opprint{PGCD} $.%
}%
	
	\begin{document}
	%\maketitle
	
	\noindent\stagiaire\hfill\datedevoir
	
	\bigskip
	\bigskip
	\begin{center}
	{\large\bfseries\titre}
	\end{center}
	\bigskip
	\bigskip
	
	%%%%%%%%%%%%%%%%%%%%%%%%%%%%%%%%%%%%%%%%%%%%%%%%%%%%%%%%%%%%%%%%%%%%%%%%%%%%%
	%%%%%%%%%%%%%%%%%%%%%%%%%%%%%%%%%%%%%%%%%%%%%%%%%%%%%%%%%%%%%%%%%%%%%%%%%%%%%
	%%%%%%%%%%%%%%%%%%%%%%%%%%%%%%%%%%%%%%%%%%%%%%%%%%%%%%%%%%%%%%%%%%%%%%%%%%%%%
	
	%%%%%%%%%%%%%%%%%%%%%%%%%%%%%%%%%%%%%%%%%%%%%%%%%%%%%%%%%%%%%%%%%%%%%%%%%%%%%
	%%%%%%%%%%%%%%%%%%%%%%%%%%%%%%%%%%%%%%%%%%%%%%%%%%%%%%%%%%%%%%%%%%%%%%%%%%%%%
	%%%%%%%%%%%%%%%%%%%%%%%%%%%%%%%%%%%%%%%%%%%%%%%%%%%%%%%%%%%%%%%%%%%%%%%%%%%%%
	\exercice 
Montrer que pour tout entier naturel $n$, $\frac{5^{n+1}+6^{n+1}}{5^n+6^n}$ est irréductible.
	
On suppose,  $\frac{5^{n+1}+6^{n+1}}{5^n+6^n}$ réductible. Il existe les entiers naturels $a,b, d$ tel que :

\begin{align}
	\frac{5^{n+1}+6^{n+1}}{5^n+6^n} &= \frac{d}{d}\times \frac{a}{b} \notag
\end{align}

$5^{n+1}+6^{n+1} =ad $ et $5^n+6^n=bd $ avec $d=pgcd(5^{n+1}+6^{n+1},5^n+6^n)$ et $a$ et $b$ premiers entre eux.

D'après le théorème de Bezout, $a$ et $b$ sont premiers entre eux si et seulement si il existe $u$ et $v \in \Z^2$ tel que $au+bv=1$. 

\begin{align}
	au+bv=1 \implies (5^{n+1}+6^{n+1})u + (5^n+6^n)v &=d \notag \\
	5(5^nu+5^{n-1}v)+6(6^nu+6^{n-1}v) &= d \notag
\end{align}

Soit $u'=5^nu+5^{n-1}v$ et $v'=6^nu+6^{n-1}v$, $u'$ et $v'$ sont non nuls car $\in \Z$. 

De plus, $5$ et $6$ sont premiers entre eux, donc $\exists u,v$ tel que $5u+6v=1$, donc $d=1$. Par hypothèse $d\neq 1$, donc l'hypothèse de départ est fausse. Si l'hypothèse de départ est fausse on peut donc conclure que $\frac{5^{n+1}+6^{n+1}}{5^n+6^n}$ est irréductible.

	%%%%%%%%%%%%%%%%%%%%%%%%%%%%%%%%%%%%%%%%%%%%%%%%%%%%%%%%%%%%%%%%%%%%%%%%%%%%%
	%%%%%%%%%%%%%%%%%%%%%%%%%%%%%%%%%%%%%%%%%%%%%%%%%%%%%%%%%%%%%%%%%%%%%%%%%%%%%
	%%%%%%%%%%%%%%%%%%%%%%%%%%%%%%%%%%%%%%%%%%%%%%%%%%%%%%%%%%%%%%%%%%%%%%%%%%%%%
	\exercice 
	Montrer que pour tout entier naturel $n$ non nul, $10^{10^n}\equiv 4(\mod 7)$.
	
	$10 \equiv 3 (7)$.
	
	\begin{align}
		10^1 &\equiv 3 (7) \notag \\
		10^2 &\equiv 2 (7) \notag \\
		10^3 &\equiv 6 (7) \notag \\
		10^4 &\equiv 4 (7) \notag \\
		10^5 &\equiv 5 (7) \notag \\
		10^6 &\equiv 1 (7) \notag \\
		10^{10} &\equiv 4 (7) \notag
	\end{align}


$10^{10^n}\equiv 4(7)$ pour $n=1$. On suppose la proposition vrai au rang $n$. 
	\begin{align}
		10^{10^n+1} &= 10^{10^{n^{10}}} \notag \\
			&\equiv (4(7))^{10} \notag \\
			&\equiv 4^{10}(7) \notag \\
			&\equiv 4(7) \notag
	\end{align}

La proposition est vraie au rang $n+1$, donc pour tout entier naturel n non nul $10^{10^n}\equiv 4(\mod 7)$.


%%%%%%%%%%%%%%%%%%%%%%%%%%%%%%%%%%%%%%%%%%%%%%%%%%%%%%%%%%%%%%%%%%%%%%%%%%%%%
%%%%%%%%%%%%%%%%%%%%%%%%%%%%%%%%%%%%%%%%%%%%%%%%%%%%%%%%%%%%%%%%%%%%%%%%%%%%%
%%%%%%%%%%%%%%%%%%%%%%%%%%%%%%%%%%%%%%%%%%%%%%%%%%%%%%%%%%%%%%%%%%%%%%%%%%%%%
	\exercice
	Résoudre dans $\Z$
	
	\[
	\left\{
	\begin{aligned}	
		x \equiv 5 (12) (1)\\
		x \equiv 3(35) (2)\\	
		x \equiv 8(13) (3)
		\end{aligned}	
	\right.\]
	
	%\PGCD{35}{13}
	
	Le $pgcd(35,13)=1$, donc il existe une unique solution modulo $(35 \times 13)$ pour (2) et (3).
	
On commence par traiter les équations $2$ et $3$. 
	
\[
	\left\{
	\begin{aligned}	
		x \equiv 3(35) (2)\\	
		x \equiv 8(13) (3)
		\end{aligned}	
	\right.\]	
	
De ces équations ont en déduit l'expression de $x$ suivante :

	\begin{center}
	$x=35k +3 \equiv 9k+3(13)=8(13)$
	\end{center}
	
On cherche ensuite à déterminer la valeur de $k$.
	\begin{align}
		9k &= 5(13) \notag \\
		3 \times 9k &= 3 \times 5(13) \notag \\
		k &=2(13) \notag \\
		k &= 13k'+2 \notag
	\end{align}

On remplace l'expression de $k$	dans l'équation.
	\begin{align}
		x &= 35(13k'+2)+3 \notag \\
		   &= 35 \times 13k' +73 \notag \\
		   &\equiv 73(35 \times 13) \notag
	\end{align}
	
	
Il reste à résoudre les deux questions suivantes :
	\[
	\left\{
	\begin{aligned}	
		x \equiv 73(455) (4)\\	
		x \equiv 8(13) (5)
		\end{aligned}	
	\right.\]
	
	Le $pgcd(73,5)=1$. Ainsi il existe une unique solution modulo $455 \times 13 $ pour (4) et (5).

De ces équations ont peut déduire, l'expression suivante :	
	\begin{center}
	$x=455k +73 \equiv 11k + 1 (\mod 12)=5 (12)$
	\end{center}

On cherche ensuite à déterminer la valeur de $k$.	
	\begin{align}
		11k&=4(12) \notag \\
		11 \times 11k &= 44(12) \notag \\	
		k &= 8(\mod 12) \notag \\
		k &= 12k' +8 \notag
	\end{align}

On remplace l'expression de $k$	dans l'équation.
	
	\begin{align}
		x &= 455(12k'+8)+73 \notag \\
		   &= 455 \times 12k' + 3640 + 73 \notag \\
		   &= 455 \times 12k'+3713 \notag \\
		   &\equiv 3713 (455 \times 12) \notag 
	\end{align}
	
On obtient pour unique solution des équations, $x\equiv 3713 (455 \times 12)$.
	
	
	 \end{document}
		