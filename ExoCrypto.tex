	\documentclass[a4paper,10pt]{article}
	
	\usepackage[utf8]{inputenc}
	\usepackage[T1]{fontenc}
	\usepackage[francais]{babel}
	\usepackage{amsthm}
	\usepackage{amsmath}
	\usepackage{graphics}
	\usepackage{amssymb}
	\usepackage{color}
	\usepackage[table,xcdraw]{xcolor}
	\usepackage{tikz}
	
	\usepackage{xlop}
	\usepackage{ifthen}
	\usepackage[top=2cm,bottom=2cm,left=2cm,right=2cm]{geometry}

	%\usepackage{mathsfs}
	
	%\title{Devoir 1 de Mathématiques}
	%\author{\textsc{RIMBON} Loy}
	%\date{\today}
	
	%!!!!!!!!!!!!!!!!!!!!!!!!!!!!               !!!!!!!!!!!!!!!!!!!!!!!!!!!!
	%!!!!!!!!!!!!!!!!!!!!!!!!!!!!   Exercices   !!!!!!!!!!!!!!!!!!!!!!!!!!!!
	%!!!!!!!!!!!!!!!!!!!!!!!!!!!!               !!!!!!!!!!!!!!!!!!!!!!!!!!!!
	
	\newcounter{exercice}
	\setcounter{exercice}{0}
	\newcommand{\exercice}{%
		\refstepcounter{exercice}%
		\bigskip
		\bigskip
		\noindent\textbf{Exercice \theexercice.}~%
	}
	%%%%%%%%%%%%%%%%%%%%%%%%%%%%%%%%%%%%%%%%%%%%%%%%%%%%%%%
	% styles TikZ pour les blocks
	\tikzstyle{block}=[draw,rectangle,fill=white,inner sep=0mm,minimum size=12mm]
	\tikzstyle{emptyblock}=[rectangle,fill=white,inner sep=0mm,minimum size=12mm]
	\tikzstyle{sumg}=[draw,circle,node distance=1cm,inner sep=0mm,minimum size=10mm]
	\tikzstyle{sum}=[draw,circle,node distance=1cm,inner sep=0mm,minimum size=6mm]
	\tikzstyle{input}=[coordinate]
	%%%%%%%%%%%%%%%%%%%%%%%%%%%%%%%%%%%%%%%%%%%%%%%%%%%%%%%
	%!!!!!!!!!!!!!!!!!!!                                !!!!!!!!!!!!!!!!!!!
	%!!!!!!!!!!!!!!!!!!!   Vos commandes personnelles   !!!!!!!!!!!!!!!!!!!
	%!!!!!!!!!!!!!!!!!!!                                !!!!!!!!!!!!!!!!!!!
	
	\newcommand{\stagiaire}{CNE Loy RIMBON}
	\newcommand{\datedevoir}{17 octobre 2016}
	\newcommand{\titre}{Devoir de cryptographie}
	
	\newcommand{\R}{\mathbb{R}}
	\newcommand{\C}{\mathbb{C}}
	\newcommand{\N}{\mathbb{N}}
	\newcommand{\Z}{\mathbb{Z}}
	\newcommand{\Q}{\mathbb{Q}}
	\newcommand{\LCI}{$E\times E\longrightarrow E$ ~}
	
	\renewcommand{\leq}{\leqslant}
	\renewcommand{\geq}{\geqslant}
	
	\newcounter{arret}%
\setcounter{arret}{0}%
\newcommand{\PGCD}[2]%
{%
\opcopy{#1}{a}%
\opcopy{#2}{b}%
\opcopy{#1}{A}%
\opcopy{#2}{B}%
\opgcd{A}{B}{PGCD}%
\noindent%
%Calculons par l'algorithme d'\textsc{Euclide} le PGCD des nombres $ \opprint{A} $ et $ \opprint{B} $.\\%
\whiledo{\equal{\thearret}{0}}%
{\opidiv*{a}{b}{q}{r}%
$ \opprint{a} = \opprint{b} \times \opprint{q} + \opprint{r} $\\%
\opcmp{r}{0}%
\ifopeq%
        \refstepcounter{arret}%
\fi%
\opcopy{b}{a}%
\opcopy{r}{b}}%
%Le PGCD des nombres $ \opprint{A} $ et $ \opprint{B} $ est le dernier reste non nul du procédé,
%c'est-à-dire $ \opprint{PGCD} $.%
}%
	
	\begin{document}
	%\maketitle
	
	\noindent\stagiaire\hfill\datedevoir
	
	\bigskip
	\bigskip
	\begin{center}
	{\large\bfseries\titre}
	\end{center}
	\bigskip
	\bigskip
	
	%%%%%%%%%%%%%%%%%%%%%%%%%%%%%%%%%%%%%%%%%%%%%%%%%%%%%%%%%%%%%%%%%%%%%%%%%%%%%
	%%%%%%%%%%%%%%%%%%%%%%%%%%%%%%%%%%%%%%%%%%%%%%%%%%%%%%%%%%%%%%%%%%%%%%%%%%%%%
	%%%%%%%%%%%%%%%%%%%%%%%%%%%%%%%%%%%%%%%%%%%%%%%%%%%%%%%%%%%%%%%%%%%%%%%%%%%%%
	
	%%%%%%%%%%%%%%%%%%%%%%%%%%%%%%%%%%%%%%%%%%%%%%%%%%%%%%%%%%%%%%%%%%%%%%%%%%%%%
	%%%%%%%%%%%%%%%%%%%%%%%%%%%%%%%%%%%%%%%%%%%%%%%%%%%%%%%%%%%%%%%%%%%%%%%%%%%%%
	%%%%%%%%%%%%%%%%%%%%%%%%%%%%%%%%%%%%%%%%%%%%%%%%%%%%%%%%%%%%%%%%%%%%%%%%%%%%%
	\exercice Test de primalité.
	
	\begin{enumerate}
	\item
		$90=(1011010)_2$.
	\item
	\medskip
	
		\begin{tabular}{|c|c|c|c|}
			\hline 
			\rule[-1ex]{0pt}{2.5ex}  $b$ & $x$ & $A^2$ \\ 
						
			\hline 
			\rule[-1ex]{0pt}{2.5ex}  90 & 1 & 2  \\ 
			\hline 
			\rule[-1ex]{0pt}{2.5ex} 45 & 1 & 4  \\ 
			\hline 
			\rule[-1ex]{0pt}{2.5ex}  22 & 4& 16 \\ 
			\hline 
			\rule[-1ex]{0pt}{2.5ex} 11 &4 &74  \\ 
			\hline 
			\rule[-1ex]{0pt}{2.5ex} 5 & 23  & 16 \\ 
			\hline 
			\rule[-1ex]{0pt}{2.5ex} 2 & 4 & 74 \\ 
			\hline 
			\rule[-1ex]{0pt}{2.5ex} 1 & 4 & 16 \\ 
			\hline 
			\rule[-1ex]{0pt}{2.5ex} 0  & 64 & 7 \\ 
			\hline 
			\end{tabular} 	
	
	\item
	\medskip
	$91$ est il premier?
	On utilise le petit théorème de Fermat.
	
	\begin{align}
		2^{90} \equiv 64 \pmod{91} \notag \\
		2^{91} \equiv 37 \pmod{91} \notag 
	\end{align}
		Ainsi, il existe un entier tel que $a^{91} \neq a \pmod{91}$, donc $91$ n'est pas un nombre premier.
		
	\item
	$2^{10}=1024 \thickapprox 10^3$.
	$2^90=2^{10^9} \thickapprox 10^{3^9 }\thickapprox10^{27} $.
	
	Il y'a environ $27$ chiffres décimaux.
	
	\end{enumerate}
	

	%%%%%%%%%%%%%%%%%%%%%%%%%%%%%%%%%%%%%%%%%%%%%%%%%%%%%%%%%%%%%%%%%%%%%%%%%%%%%
	%%%%%%%%%%%%%%%%%%%%%%%%%%%%%%%%%%%%%%%%%%%%%%%%%%%%%%%%%%%%%%%%%%%%%%%%%%%%%
	%%%%%%%%%%%%%%%%%%%%%%%%%%%%%%%%%%%%%%%%%%%%%%%%%%%%%%%%%%%%%%%%%%%%%%%%%%%%%
	\exercice Exponentiation rapide.
	
	Soit $b=b_l....b_1b_0$. 
	
	\begin{enumerate}
	\item
	Exprimer $l$ en fonction de la valeur de $\log b$.
	
	$l=\log b$.

	$b$ à une longeur de $l+1$ bits. 
	
	\item
	La ligne 4 de l'algorithme exponentiation consiste à élever la variable $a$ au carré. Cette ligne est exécutée à chaque  tour de l'algorithme, excepté au dernier tour au l'élévation au carrée est inutile. La ligne est donc exécuté $l$ fois.
	
	\item
	La multiplication de la ligne 3, est  des que $b$ est impair, ou encore des qu'un bit de la représentation binaire de $b$ est à 1. Donc si $b$ à $k$ bits à 1, la ligne 3 sera exécutée $k$ fois. 
	
	\item
	La valeur maximale du nombre de multiplication est obtenue avec un nombre $b$ ayant tout ces bits à $1$. Dans ce cas la il y'aura :
	
	$V_{max}=2l+1$ multiplications.
	
	La valeur moyenne est obtenue en divisant par 2 la valeur maximale et minimale du nombre de multiplication :
	
	$V_{moy}=\frac{2l+1+l}{2} =\frac{3l+1}{2}$
	
	
	\end{enumerate}
%%%%%%%%%%%%%%%%%%%%%%%%%%%%%%%%%%%%%%%%%%%%%%%%%%%%%%%%%%%%%%%%%%%%%%%%%%%%%
%%%%%%%%%%%%%%%%%%%%%%%%%%%%%%%%%%%%%%%%%%%%%%%%%%%%%%%%%%%%%%%%%%%%%%%%%%%%%
%%%%%%%%%%%%%%%%%%%%%%%%%%%%%%%%%%%%%%%%%%%%%%%%%%%%%%%%%%%%%%%%%%%%%%%%%%%%%
	\exercice Méthode $\rho$ de Pollard.
	
	$N=91$
	$f(x)=x^2+1 \pmod {91}$
	
	\begin{align}
		x_0 &= 0 \notag \\
		x_1 &= f(x_0)=0^2+1 \equiv 1 \pmod{91}\notag \\
		x_2 &= f(x_1)=1^2+1 \equiv 2 \pmod{91}\notag \\
		x_3 &= f(x_2)=2^2+1 \equiv 5 \pmod{91}\notag \\
		x_4 &= f(x_3)=5^2+1 \equiv 26 \pmod{91}\notag \\
		x_5 &= f(x_4)=26^2+1 \equiv 40 \pmod{91}\notag \\
		x_6 &= f(x_5)=0^2+1 \equiv 54 \pmod{91}\notag \\
	\end{align}		
	
	\begin{align}
		pgcd(x_1-x_2,91) &= pgcd(1,91)=1 \notag
	\end{align}	
%	\PGCD{91}{24}
	\begin{align}	
		pgcd(x_2-x_4,91) &= pgcd(24,91)= pgcd(91,24)\notag \\
		91 &= 24 \times 3 + 19 \notag \\
		24 &= 19 \times 1 + 5 \notag \\
		19 &= 5 \times 3 + 4 \notag \\
		5 &= 4 \times + 1 \notag \\
		4 &= 4 \times 1 + 0 \notag
	\end{align}
		$pgcd(x_2-x_4,91)=1$
	\begin{align}	
		pgcd(x_3-x_6,91) &= pgcd(42,91)= pgcd(91,42)\notag \\
		91 &= 42 \times 2 + 7 \notag \\
		42 &= 7 \times 6 + 0\notag
	\end{align}
		$pgcd(x_3-x_6,91) = 7$
	Donc $7$ est un diviseur de $91$, $91=7 \times 13$.

%%%%%%%%%%%%%%%%%%%%%%%%%%%%%%%%%%%%%%%%%%%%%%%%%%%%%%%%%%%%%%%%%%%%%%%%%%%%%
%%%%%%%%%%%%%%%%%%%%%%%%%%%%%%%%%%%%%%%%%%%%%%%%%%%%%%%%%%%%%%%%%%%%%%%%%%%%%
%%%%%%%%%%%%%%%%%%%%%%%%%%%%%%%%%%%%%%%%%%%%%%%%%%%%%%%%%%%%%%%%%%%%%%%%%%%%%
	\exercice Algorithme d'Euclide.
	\begin{enumerate}
	
	\item
	\begin{align}
		pgcd(91,35) &= pgcd(35,21) \notag \\
			&= pgcd(21,14) \notag \\
			&= pgcd(14,7) \notag \\
			&= pgcd(7,0) \notag \\
			&=7 \notag
	\end{align}
	
	\item
	Calculer par l'algorithme d'Euclide étendue l'inverse de 11 modulo 91.
	
	\begin{align}
		91 &= 11 \times 8 + 3 \notag \\
		11 &= 3 \times 3 + 2 \notag \\
		3 &= 2 \times 1 + 1 \notag \\
		2 &= 1 \times 2 + 0 \notag 
	\end{align}
	
	\begin{align}
		1 &= 3-2 \notag \\
			&= 3-(11-3 \times 3) \notag \\
			&= 3 \times 4 - 11 \notag \\
			&=(91 -11 \times 8)\times 4-11 \notag \\
			&= 91 \times 4 -11 \times 32-11 \notag \\
			&= 91 \times 4-11 \times 33 \notag
	\end{align}
	
	Ainsi, $91 \times 4-11\times 33=1$.
	
	\begin{align}
		91 \times 4-11\times 33 &=1 \notag \\
		11\times 33 &= 91 \times 4-1 \notag \\
		11 &= \frac{91 \times 4-1}{33} \notag \\
		11^{-1} &= 91 \times \frac{4}{33}- \notag
	\end{align}
	
	$11^{-1}\equiv -33\pmod{91}\equiv 58\pmod{91}$
	
	\item
	Montrer que $\forall n$, les nombres de Fibonacci $f_n$ et $f_{n+1}$ sont premiers entre eux.
	
	$f_0=1 f_1=1 $ et $f_{n+2}=f_{n+1}+f_n$
	
	$f_2 $et $f_1$ sont premiers entre eux. On suppose la proposition vrai au rang n, $pgcd(f_{n-1},f_n)=1$.
	
	$f_{n+1} = f_n+f_{n-1}\Longleftrightarrow a=bq+r$, avec $q=1$ et $r=f_{n-1}$.
	
	Ainsi,
	\begin{align}
		pgcd(f_{n+1},f_n)&=pgcd(f_n, f_{n+1} mod f_n) \notag \\
			&=pgcd(f_n,f_{n-1})\notag \\
			&= 1\notag
	\end{align}
	
	
	\end{enumerate}	
	%%%%%%%%%%%%%%%%%%%%%%%%%%%%%%%%%%%%%%%%%%%%%%%%%%%%%%%%%%%%%%%%%%%%%%%%%%%%%
%%%%%%%%%%%%%%%%%%%%%%%%%%%%%%%%%%%%%%%%%%%%%%%%%%%%%%%%%%%%%%%%%%%%%%%%%%%%%
%%%%%%%%%%%%%%%%%%%%%%%%%%%%%%%%%%%%%%%%%%%%%%%%%%%%%%%%%%%%%%%%%%%%%%%%%%%%%
	\exercice	 Menteurs de MILLER-RABIN pour $N=561$
	
	\begin{enumerate}
	\item
	Soit $(b_1,b_2,b_3,b_4)=(a^{35},a^{70},a^{140},a^{280}$
	Un menteur, est un élément qui indique qu'un nombre est premier alors que ce n'est pas le cas. C'est à dire : $b_i\equiv \pm 1\pmod{561}$
	\item
		Soit $x^2\equiv -1\pmod{561}$. Montrer que $x^2\equiv 2\pmod {3}$
		\begin{align}
			x^2 &= -1 + 561k \notag \\
			x^2 +1 &= 561k \notag \\
			x^2+1 &= 187 \times 3k \notag \\
			x^2 &= -1+187 \times 3k \notag \\
			x^2 &\equiv -1\pmod{3} \notag \\
			x^2 &\equiv 2\pmod{3} \notag
		\end{align}
		
		$b_2=b_1^2$, $b_3=b_2^2$ et $b_4=b_3^2$
		
		\begin{align}
			1^2 &\equiv 1 \pmod{3} \notag \\
			2^2 &\equiv 1 \pmod{3} \notag \\
			3^2 &\equiv 0 \pmod{3} \notag \\
			4^2 &\equiv 1 \pmod{3} \notag \\
			5^2 &\equiv 1 \pmod{3} \notag \\
			6^2 &\equiv 0 \pmod{3} \notag
		\end{align}				
		On suppose, qu'un carré n'est jamais congru à $2\pmod{3}$. $n^2 \neq 2\pmod{3}$.
		
		$(n+1)^2=n^2+1^2$, car nous sommes dans l'anneau $\Z/2\Z$.
		De plus, on sait que $n^2 \neq 2\pmod{3}$ et $1^2 \equiv 1 \pmod{3}$.
		Donc $(n+1)^2$ n'est pas congru à  $2\pmod{3}$. Ainsi il est impossible que $b_2,b_3,b_4$ soient égal à -1.
		
		
	
	\item
	Donner une condition pour que $a^{35}$ soit un menteur.
	$a$ est un menteur, c'est à dire qu'il permet d'affirmer que $n$ est un nombre premier, si il vérifie la condition suivante :
	$a^{35}\equiv \pm1 \pmod{561}$.
	\item
	$x^{35}=x^5 \times x^{30}=x^5 \times (x^{10})^3$.
	
	Or, $\varphi(11)=10$ et $a^{\varphi(11)}\equiv 1 \pmod{11}$.
	Donc $(x^{10})^3\equiv 1\pmod{11}$ et $x^{35}\equiv x^5 \pmod{11}$.
	
	$x^{35}=x^3 \times x^{16}$.
	
	Or, $\varphi(17)=16$ et $a^{\varphi(17)}\equiv 1 \pmod{17}$.
	Donc $(x^{16})^3\equiv 1\pmod{17}$ et $x^{35}\equiv x^3 \pmod{17}$.
	
	$x^{35}=x \times x^{2^{17}}$.
	
	Or, $\varphi(3)=2$ et $a^{\varphi(3)}\equiv 1 \pmod{3}$.
	Donc $(x^{2^{17}})\equiv 1\pmod{3}$ et $x^{35}\equiv x \pmod{3}$.
	
	$x^{35} \equiv 1 \pmod{561}$ équivaut donc aux trois équations :
	\begin{center}
	$x \equiv 1\pmod{3}$, $  x^5\equiv 1 \pmod{11}$ et $x^3\equiv 1 \pmod{17}$
		\end{center}
	\item
	\item
	\end{enumerate}
%%%%%%%%%%%%%%%%%%%%%%%%%%%%%%%%%%%%%%%%%%%%%%%%%%%%%%%%%%%%%%%%%%%%%%%%%%%%%
%%%%%%%%%%%%%%%%%%%%%%%%%%%%%%%%%%%%%%%%%%%%%%%%%%%%%%%%%%%%%%%%%%%%%%%%%%%%%
%%%%%%%%%%%%%%%%%%%%%%%%%%%%%%%%%%%%%%%%%%%%%%%%%%%%%%%%%%%%%%%%%%%%%%%%%%%%%
	\exercice Nombres de Fibonacci (1)

On considère la suite Fibonacci, décrite par $f_n=f_{n-1}+f_{n-2}$, $f_0=0$, $f_1=1$.

	\begin{enumerate}
	\item
		Pour un nombre $n$, sa taille $l$ est égale à $log_2(n)$.
	\item
	Montrer que le calcul de $fibo1(n)$ demande $f_{n+1}-1 $ additions.
	
	Pour $f_0$, on bien $0$ addition, ce qui correspond bien à : $f_1-1=1-1=0$.
	Idem pour $f_1$, $f_2-1=1-1=0$.
	
	On suppose la proposition vrai au rang $n$. Pour, $fibo1(n)$, $f_n+1-1 $ additions sont nécessaires.
	
	On cherche à vérifier la proposition au rang $n+1$.
	
	$f_{n+1}=f_n+f_{n-1}$.
	
	$fibo1(n+1)=\underbrace{fibo1(n)}_{f_{n+1}-1 additions}\underbrace{+}_{1 addition}\underbrace{fibo1(n-1)}_{f_n-1n additions}$.
	
	Nombre d'additions de $fibo1(n+1)= f_{n+1}+f_n-1 +1- 1$
	
	
	Il faut donc : $f_{n+2}+1$ additions pour $f_{n+1}$. La proposition est donc vérifiée au rang n+1.
	
	
	\item
	
	
	\end{enumerate}

%%%%%%%%%%%%%%%%%%%%%%%%%%%%%%%%%%%%%%%%%%%%%%%%%%%%%%%%%%%%%%%%%%%%%%%%%%%%%
%%%%%%%%%%%%%%%%%%%%%%%%%%%%%%%%%%%%%%%%%%%%%%%%%%%%%%%%%%%%%%%%%%%%%%%%%%%%%
%%%%%%%%%%%%%%%%%%%%%%%%%%%%%%%%%%%%%%%%%%%%%%%%%%%%%%%%%%%%%%%%%%%%%%%%%%%%%
	\exercice Nombres de Fibonacci (2)

$F_i=\begin{pmatrix} 
f_i \\
f_{i+1} 
\end{pmatrix}
\quad$	
	
	\begin{enumerate}
	\item
		$F_0=\begin{pmatrix} 
					0 \\
					1 
				\end{pmatrix}
				\quad$	
		$F_1=\begin{pmatrix} 
					1 \\
					1 
				\end{pmatrix}
				\quad$
		$F_2=\begin{pmatrix} 
					1 \\
					2 
				\end{pmatrix}
				\quad$	
	\item
	Si $n=0$ la fonction renvoi $F_0$, sinon on calcul $F_{n-1}$  qui sera renvoyé dans les variables $u$ et $v$. $F_{i+1}$ est calculé à partir de $u$ et $v$. Le vecteur final renvoyé correspond bien à $F_i$.
	\item
	La fonction fibo2 est une fonction récursive. Pour un nombre $n$ elle sera exécutée $n$ fois. Une seule addition de 2 entiers est réalisé à chaque  appel de la fonction lors de la dernière instruction $(u+v)$.
Pour un élément $F_i$ de la suite la dernière addition sera donc $f_i+f_{i-1}$. Si $f_i$ est un nombre de $l$ bits. La dernière addition comportera donc $l$ opérations.


	\item
	
	\end{enumerate}
%%%%%%%%%%%%%%%%%%%%%%%%%%%%%%%%%%%%%%%%%%%%%%%%%%%%%%%%%%%%%%%%%%%%%%%%%%%%%
%%%%%%%%%%%%%%%%%%%%%%%%%%%%%%%%%%%%%%%%%%%%%%%%%%%%%%%%%%%%%%%%%%%%%%%%%%%%%
%%%%%%%%%%%%%%%%%%%%%%%%%%%%%%%%%%%%%%%%%%%%%%%%%%%%%%%%%%%%%%%%%%%%%%%%%%%%%
	\exercice Nombres de Fibonacci (3)
	On reprend la suite $F_i$ de l'exercice prédédent.
	
	\begin{enumerate}
	\item
		Montrer que les vecteurs $F_i$ satisfont la relation suivante :
		
		$F_{i+1}=\begin{pmatrix}
						0 & 1 \\		
						1 & 1
						\end{pmatrix}F_i$
		
		$F_{i+1}=\begin{pmatrix}
						0 & 1 \\		
						1 & 1
						\end{pmatrix}\times
						\begin{pmatrix} 
							f_i \\
							f_{i+1} 
						\end{pmatrix}
					=\begin{pmatrix}
						0 \times f_i+1 \times f_{i+1}\\		
						1 \times f_i+1 \times f_{i+1}
						\end{pmatrix}
					=\begin{pmatrix}
						f_{i+1}\\		
						f_i+f_{i+1}
						\end{pmatrix}
					=\begin{pmatrix}
						f_{i+1}\\		
						f_{i+2}
						\end{pmatrix}$
			
		La relation est donc vérifiée.			
	\item
	Donner un algorithme polynomial de calcul du terme $f_n$ de la suite de Fibonacci.
	
	%F_n=
	\item
	
	\end{enumerate}		
	
	
		
	
	 \end{document}
		